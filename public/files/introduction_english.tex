\newif\ifanswer\answertrue
\answerfalse

\documentclass[10pt]{jsarticle}
\usepackage[margin=15truemm]{geometry}
\usepackage[dvipdfmx]{graphicx}
\usepackage{ascmac} % for screen
\usepackage{subfigure} % for subfigure
\usepackage{multicol}
\usepackage{amsmath}
\usepackage{mathtools}
\usepackage{amssymb}
\usepackage{multicol}
\usepackage{tikz}
\usepackage{here}
\usepackage{setspace}
\usetikzlibrary{intersections, calc, arrows.meta}
\usepackage{fancyhdr}
\usepackage{wrapfig}
\pagestyle{fancy}

\ifanswer
\lhead{英語 導入 解答}
\else
\lhead{英語 導入}
\fi
\rhead{\number\month\number\day}

\ifanswer
\newcommand{\answer}[2]{{\color{orange}#2}}
\newcommand{\page}[1]{#1}
\newcommand{\question}[2]{{\color{orange}#2}}
\else
\newcommand{\answer}[2]{\vspace{#1mm}}
\newcommand{\page}[1]{}
\newcommand{\question}[2]{#1}
\fi%answer

\renewcommand{\thesubsubsection}{(\arabic{subsubsection})}
%\renewcommand{\baselinestretch}{2}

\begin{document}



\section{英語の語順}
\begin{screen}
  \vspace{1cm}
\end{screen}

\section{動詞の変化について}
\begin{itembox}[l]{be動詞5、使い分け}
  \begin{multicols}{5}
    \begin{itemize}
      \item  \item \item \item \item
    \end{itemize}
  \end{multicols}
  \vspace{3mm}
  \begin{itemize}
    \item 肯定文 \answer{5}{}
    \item 疑問文 \answer{5}{}
    \item 否定文 \answer{5}{}
  \end{itemize}
\end{itembox}

\begin{itembox}[l]{一般動詞}

  \begin{itemize}
    \item 肯定文 \answer{5}{}
    \item 疑問文 \answer{5}{}
    \item 否定文 \answer{5}{}
  \end{itemize}
  時制や主語による変化とそれに対するdoの変化
  \begin{multicols}{3}
    \begin{itemize}
      \item 原形 do
      \item \answer{5}{}
      \item \answer{5}{}
    \end{itemize}
  \end{multicols}
\end{itembox}

\section{助動詞}
\begin{itembox}[l]{基本形}
  \vspace{1cm}
\end{itembox}
\begin{itembox}[l]{例}
  \begin{multicols}{3}
    \begin{itemize}
      \item \vspace{10mm}
      \item \vspace{10mm}
      \item \vspace{10mm}
    \end{itemize}
  \end{multicols}
  \vspace{10mm}
\end{itembox}

\newpage

\section{人称代名詞}
 {\renewcommand\arraystretch{2}
  \begin{table}[htbp]
    \centering
    \begin{tabular}{|c|p{3cm}|p{3cm}|p{3cm}|p{3cm}|}
      \hline
                 & 主格 & 所有格 & 目的格 & 所有代名詞 \\
      \hline
      意味       &      &        &        &            \\
      \hline
      使う場面   &      &        &        &            \\
      \hline
      \hline
      私         &      &        &        &            \\
      \hline
      あなた     &      &        &        &            \\
      \hline
      彼         &      &        &        &            \\
      \hline
      彼女       &      &        &        &            \\
      \hline
      それ       &      &        &        & なし       \\
      \hline
      私たち     &      &        &        &            \\
      \hline
      あなたたち &      &        &        &            \\
      \hline
      彼ら       &      &        &        &            \\
      \hline
    \end{tabular}
  \end{table}
 }

\section{数字}
\begin{screen}
  \begin{multicols}{4}
    \begin{itemize}
      \item 1 \answer{5}{one}
      \item 2 \answer{5}{two}
      \item 3 \answer{5}{three}
      \item 4 \answer{5}{four}
      \item 5 \answer{5}{five}
      \item 6 \answer{5}{six}
      \item 7 \answer{5}{seven}
      \item 8 \answer{5}{eight}
      \item 9 \answer{5}{nine}
      \item 10 \answer{5}{ten}
      \item 11 \answer{5}{eleven}
      \item 12 \answer{5}{twelve}
      \item 13 \answer{5}{thirteen}
      \item 14 \answer{5}{fourteen}
      \item 15 \answer{5}{fifteen}
      \item 16 \answer{5}{sixteen}
      \item 17 \answer{5}{seventeen}
      \item 18 \answer{5}{eighteen}
      \item 19 \answer{5}{nineteen}
      \item 20 \answer{5}{twenty}
      \item 21 \answer{5}{twenty-one}
      \item 30 \answer{5}{thirty}
      \item 40 \answer{5}{forty}
      \item 50 \answer{5}{fifty}
      \item 60 \answer{5}{sixty}
      \item 70 \answer{5}{seventy}
      \item 80 \answer{5}{eighty}
      \item 90 \answer{5}{ninety}
      \item 100 \answer{5}{one hundred}
      \item 1000 \answer{5}{one thousand}
    \end{itemize}
  \end{multicols}
\end{screen}


\newpage

\section{序数}
\begin{screen}
  \begin{multicols}{4}
    \begin{itemize}
      \item 1番目 \answer{5}{first}
      \item 2番目 \answer{5}{second}
      \item 3番目 \answer{5}{third}
      \item 4番目 \answer{5}{fourth}
      \item 5番目 \answer{5}{fifth}
      \item 6番目 \answer{5}{sixth}
      \item 7番目 \answer{5}{seventh}
      \item 8番目 \answer{5}{eighth}
      \item 9番目 \answer{5}{ninth}
      \item 10番目 \answer{5}{tenth}
      \item 11番目 \answer{5}{eleventh}
      \item 12番目 \answer{5}{twelfth}
      \item 13番目 \answer{5}{thirteenth}
      \item 20番目 \answer{5}{twentieth\\}
      \item 21番目 \answer{5}{\\twenty-first}
      \item 30番目 \answer{5}{thirtieth}
    \end{itemize}
  \end{multicols}
\end{screen}

\section{季節}
\begin{screen}
  \begin{multicols}{4}
    \begin{itemize}
      \item 春 \answer{5}{spring}
      \item 夏 \answer{5}{summer}
      \item 秋 \answer{5}{fall}
      \item 冬 \answer{5}{winter}
    \end{itemize}
  \end{multicols}
\end{screen}

\section{月}
\begin{screen}
  \begin{multicols}{4}
    \begin{itemize}
      \item 1月 \answer{5}{January}
      \item 2月 \answer{5}{February}
      \item 3月 \answer{5}{March}
      \item 4月 \answer{5}{April}
      \item 5月 \answer{5}{May}
      \item 6月 \answer{5}{June}
      \item 7月 \answer{5}{July}
      \item 8月 \answer{5}{August}
      \item 9月 \answer{5}{September}
      \item 10月 \answer{5}{October}
      \item 11月 \answer{5}{November}
      \item 12月 \answer{5}{December}
    \end{itemize}
  \end{multicols}
\end{screen}


\section{曜日}
\begin{screen}
  \begin{multicols}{4}
    \begin{itemize}
      \item 月曜日 \answer{5}{Monday}
      \item 火曜日 \answer{5}{Tuesday}
      \item 水曜日 \answer{5}{Wednesday}
      \item 木曜日 \answer{5}{Thursday}
      \item 金曜日 \answer{5}{Friday}
      \item 土曜日 \answer{5}{Saturday}
      \item 日曜日 \answer{5}{Sunday}
    \end{itemize}
  \end{multicols}
\end{screen}

\section{日・週・月・年}
\begin{screen}
  \begin{multicols}{4}
    \begin{itemize}
      \item 今日 \answer{5}{today}
      \item 昨日 \answer{5}{yesterday}
      \item 明日 \answer{5}{tomorrow}
      \item 今週 \answer{5}{this week}
      \item 先週 \answer{5}{last week}
      \item 来週 \answer{5}{next week}
      \item 今月 \answer{5}{this month}
      \item 先月 \answer{5}{last month}
      \item 来月 \answer{5}{next month}
      \item 今年 \answer{5}{this year}
      \item 去年 \answer{5}{last year}
      \item 来年 \answer{5}{next year}
    \end{itemize}
  \end{multicols}
\end{screen}

\section{日付・時間}
\begin{screen}
  \begin{itemize}
    \item 今何時ですか? \answer{5}{}
    \item 10時です。 \answer{5}{}
    \item あなたは何時に起きますか? \answer{5}{}
    \item 6時に起きます。 \answer{5}{}
    \item 今日は何月何日ですか? \answer{5}{}
    \item 10月11日です。 \answer{5}{}
    \item 今日は何曜日ですか? \answer{5}{}
    \item 木曜日です。 \answer{5}{}
  \end{itemize}
\end{screen}

\section{天気}
\begin{itembox}[l]{天気}
  \begin{multicols}{4}
    \begin{itemize}
      \item 晴れの \answer{5}{}
      \item くもりの \answer{5}{}
      \item 雨の \answer{5}{}
      \item 雪の \answer{5}{}
    \end{itemize}
  \end{multicols}
\end{itembox}
\begin{itembox}[l]{寒暖}
  \begin{multicols}{4}
    \begin{itemize}
      \item 寒い \answer{5}{}
      \item 涼しい \answer{5}{}
      \item 暖かい \answer{5}{}
      \item 暑い \answer{5}{}
    \end{itemize}
  \end{multicols}
\end{itembox}
\begin{itembox}[l]{特殊なit}
  \begin{itemize}
    \item 今日は雨である。(降っているかわからない) \answer{5}{}
    \item 雨が降っている。(2)
    \item  \answer{5}{}
  \end{itemize}
\end{itembox}

\section{命令文}
\begin{screen}
  \begin{itemize}
    \item 基本形 \answer{5}{}
    \item 〜するな \answer{5}{}
    \item 〜してください \answer{5}{}
    \item 〜しましょう \answer{5}{}
    \item ex.親切にしなさい \answer{5}{}
  \end{itemize}
\end{screen}
\section{感嘆文}
\begin{screen}
  \begin{itemize}
    \item  \answer{5}{}
    \item  \answer{5}{}
  \end{itemize}
\end{screen}
\section{疑問詞}
\begin{screen}
  \begin{multicols}{4}
    \begin{itemize}
      \item  \answer{5}{}
      \item  \answer{5}{}
      \item  \answer{5}{}
      \item  \answer{5}{}
      \item  \answer{5}{}
      \item  \answer{5}{}
      \item  \answer{5}{}
      \item  \answer{5}{}
    \end{itemize}
  \end{multicols}
  ex.
  \begin{multicols}{2}
    \begin{itemize}
      \item どのバス \answer{5}{}
      \item なんの食べ物 \answer{5}{}
    \end{itemize}
  \end{multicols}
\end{screen}

\newpage

\subsubsection*{前置詞}

{\renewcommand\arraystretch{1.8}
  \begin{table}[H]
    \centering
    \begin{tabular}{|c|p{2cm}||c|p{2cm}||c|p{2cm}|}
      \hline
      意味       & 単語 & 意味           & 単語 & 意味             & 単語 \\ \hline\hline
      〜の上に   &      & 〜で、〜に     &      & 〜の間に(時間) &      \\ \hline
      〜の下に   &      & 〜といっしょに &      & 〜の間に(時間) &      \\ \hline
      〜の中に   &      & 〜の           &      & 〜の間に(場所) &      \\ \hline
      〜の中へ   &      & 〜のために     &      & 〜の後に         &      \\ \hline
      〜の近くに &      & 〜によって     &      & 〜の前に         &      \\ \hline
      〜のそばに &      & 〜のように     &      & 〜について       &      \\ \hline
      〜から     &      & 〜にとって     &      & 〜まで           &      \\ \hline
      〜へ       &      & 〜なしで       &      & 〜までに         &      \\ \hline
    \end{tabular}
  \end{table}
}

\subsubsection*{接続詞}


{\renewcommand\arraystretch{1.8}
  \begin{table}[H]
    \centering
    \begin{tabular}{|c|p{2cm}||c|p{2cm}||c|p{2cm}|}
      \hline
      意味   & 単語 & 意味         & 単語 & 意味     & 単語 \\ \hline\hline
      〜と   &      & もし〜ならば &      & 〜の前に &      \\ \hline
      しかし &      & 〜の間に     &      & 〜の後に &      \\ \hline
      しかし &      & 〜の時       &      & 〜だが   &      \\ \hline
      〜か   &      & なぜなら     &      &          &      \\ \hline
    \end{tabular}
  \end{table}
}

\begin{itembox}[l]{接続詞と前置詞の違い}
  \vspace{1cm}
\end{itembox}


\end{document}