\newif\ifanswer\answertrue
\answerfalse

\documentclass[10pt]{jsarticle}
\usepackage[margin=15truemm]{geometry}
\usepackage[dvipdfmx]{graphicx}
\usepackage{ascmac} % for screen
\usepackage{subfigure} % for subfigure
\usepackage{multicol}
\usepackage{amsmath}
\usepackage{mathtools}
\usepackage{amssymb}
\usepackage{multicol}
\usepackage{tikz}
\usepackage{here}
\usepackage{setspace}
\usepackage{wrapfig}
\usepackage{setspace}
\usepackage{fancyhdr}
\usepackage{xcolor}
\pagestyle{fancy}
\ifanswer
\lhead{英語 高校 解答}
\else
\lhead{英語 高校}
\fi
\rhead{\number\month\number\day}

\ifanswer
\newcommand{\answer}[2]{{\color{orange}#2}}
\newcommand{\page}[1]{#1}
\newcommand{\question}[2]{{\color{orange}#2}}
\else
\newcommand{\answer}[2]{\vspace{#1mm}}
\newcommand{\page}[1]{}
\newcommand{\question}[2]{#1}
\fi%answer

\begin{document}

\newpage
\section{時制}
\begin{itembox}[l]{be動詞5、使い分け、with過去分詞・現在分詞}
	\answer{15}{
		\begin{multicols}{4}
			\begin{description}
				\item[原形] be
				\item[現在] is, am, are
				\item[過去] was, were
				\item[過去分詞] been
			\end{description}
		\end{multicols}

		現在分詞と一緒で進行形、過去分詞と一緒で受動態
	}
\end{itembox}

\begin{itembox}[l]{現在完了形(表現、意味3、よく使う副詞とその意味9)}
	\answer{20}{
		現在を基準として時間の幅があることがポイント
		\begin{itemize}
			\item 〜したところだ just(肯定文), yet(否定文、疑問文), already(肯定文)
			\item 〜したことがある ever, never, once, twice, three times, many times, How many times 疑問文?, How often 疑問文?
			\item 〜し続けている for 期間, since 基準, How long 疑問文?
		\end{itemize}
	}
\end{itembox}
\begin{itembox}[l]{現在完了形と過去完了形の違い}
	\answer{10}{基準が現在なのか過去なのか}
\end{itembox}
\begin{itembox}[l]{完了形と一緒に使えないもの}
	\answer{10}{lately/recently,so far(今までのところ),now(たった今) yesterday,last week[month,year],~ago,in2009, When~(いつ〜したか) when(〜だったとき) }
\end{itembox}

\begin{itembox}[l]{感嘆文(意味、表現2、使い分け)}
	\answer{10}{
		\begin{itemize}
			\item How 形 S V!
			\item What a 形 名 S V!
		\end{itemize}
	}
\end{itembox}

\section{受け身}

\begin{itembox}[l]{意味、表現、よく使う前置詞とその意味}
	\answer{10}{〜される、be 過去分詞、by(〜によって)}
\end{itembox}

\begin{itembox}[l]{重要表現}
	\begin{multicols}{2}
		\begin{itemize}
			\item 〜に興味がある \answer{5}{be interested in}
			\item 〜に驚く \answer{5}{be surprised at(by)}
			\item 〜で覆われている \answer{5}{be covered with}
			\item 〜に話しかけられる \answer{5}{be spoken to by}
			\item 〜に知られている \answer{5}{be known to}
			\item 〜で知られている \answer{5}{be known for}
			\item 〜に満足する \answer{5}{be satisfied with}
			\item 〜でいっぱいだ \answer{5}{be filled with}
			\item 〜を心配する \answer{5}{be worried about}
			\item 〜に喜ぶ \answer{5}{be pleased with}
			\item 〜に失望する \answer{5}{be disappointed with(at)}
			\item 〜でケガをする \answer{5}{be injured in}
			\item 〜に笑われた \answer{5}{be laughed at}
			\item 〜だそうだ \answer{5}{It is said that }
		\end{itemize}
	\end{multicols}
\end{itembox}

\newpage

\section{文型}
\vspace{-10mm}

\begin{multicols}{2}
	\subsubsection*{第1文型}
	\begin{itembox}[l]{形}
		\answer{6}{S V}
	\end{itembox}


	\subsubsection*{第3文型}
	\begin{itembox}[l]{形}
		\answer{6}{S V O}
	\end{itembox}

\end{multicols}
\vspace{-5mm}

\subsubsection*{第2文型}
\begin{itembox}[l]{形、関係、動詞の例}
	\answer{6}{
		S V C、S=C
		\begin{itemize}
			\item be動詞
			\item 〜のままである remain (残る)/keep/stay
			\item 〜のようだ look/seem(思われる)/appear(現れる) *seem = look + sound
			\item 〜になる become/get/grow/turn
			\item 〜の感じがする feel/smell/taste/sound
		\end{itemize}
	}
\end{itembox}
\question{
	\begin{itembox}[l]{動詞}
		\begin{multicols}{3}
			\begin{itemize}
				\item look \answer{5}{}
				\item sound \answer{5}{}
				\item seem \answer{5}{}
				\item taste \answer{5}{}
				\item keep \answer{5}{}
				\item become \answer{5}{}
				\item smell \answer{5}{}
				\item get \answer{5}{}
			\end{itemize}
		\end{multicols}
	\end{itembox}
}{}

\vspace{-5mm}
\subsubsection*{第4文型}
\begin{itembox}[l]{形、第3文型への書き換え}
	\answer{6}{
		S V O1 O2 = S V O2 to(for, of) O1
	}
\end{itembox}
\begin{itembox}[l]{動詞}
	\question{

		\begin{multicols}{3}
			\begin{itemize}
				\item give \answer{5}{}
				\item buy \answer{5}{}
				\item make \answer{5}{}
				\item show \answer{5}{}
				\item teach \answer{5}{}
				\item send \answer{5}{}
				\item cook \answer{5}{}
				\item tell \answer{5}{}
				\item ask \answer{5}{}
			\end{itemize}
		\end{multicols}
		*書き換える時の前置詞3、どれがどれか
		\vspace{3mm}
	}{
		目の前に相手が必ず必要ならto、いなくてもその行為ができればfor
		\begin{description}
			\item[to] give, show, send, teach, tell, lend
			\item[for] buy, make, cook, chose, get
			\item[of] ask
		\end{description}

	}
\end{itembox}

\vspace{-5mm}
\subsubsection*{第5文型}
\begin{itembox}[l]{形、関係}
	\answer{6}{S V O C 、O = C}
\end{itembox}

\begin{itembox}[l]{動詞}
	\begin{multicols}{3}
		\begin{itemize}
			\item make \answer{5}{AをBの状態にさせる}
			\item call \answer{5}{AをBと呼ぶ}
			\item name \answer{5}{AをBと名付ける}
			\item find \answer{5}{AがBとわかる}
			\item paint \answer{5}{AをBに塗る}
			\item keep \answer{5}{AをBのままにする}
			\item leave \answer{5}{AをBのままにする}
			\item elect \answer{5}{AをBにえらぶ}
			\item think \answer{5}{AをBと思う}
		\end{itemize}
	\end{multicols}
\end{itembox}


\newpage
\question{\section{助動詞 英語$\rightarrow$ 日本語}}{{\color{black}\section{助動詞}}}


\begin{itembox}[l]{意味}
	\begin{multicols}{2}
		\begin{itemize}
			\item can(2) \answer{5}{〜することができる、〜の可能性がある}
			\item may(2) \answer{5}{〜してもよい、〜かもしれない}
			\item must(2) \answer{5}{〜しなければならない、〜の違いない}
			\item used to(2) \answer{5}{かつて〜だった、よく〜したものだ}
			\item should(2) \answer{5}{〜すべき、〜のはずだ}
			\item have to \answer{5}{〜する必要がある}
			\item Shall we〜? \answer{5}{〜しませんか}
			\item will \answer{5}{〜するでしょう}
			\item be going to \answer{5}{〜するつもりだ}
			\item had better \answer{5}{〜した方が良い}
			\item ought to \answer{5}{〜すべき}
			\item Shall I〜? \answer{5}{〜しましょうか}
			\item would often \answer{5}{よく〜したものだ}
		\end{itemize}
	\end{multicols}
\end{itembox}

\begin{itembox}[l]{ought toの否定}
	\answer{10}{ought not to}
\end{itembox}


\begin{multicols}{2}
	\begin{itembox}[l]{canの書き換え}
		\answer{8}{be able to}
	\end{itembox}
	\begin{itembox}[l]{willの書き換え}
		\answer{8}{be going to}
	\end{itembox}
	\begin{itembox}[l]{mustの書き換え}
		\answer{8}{have to}
	\end{itembox}
	\begin{itembox}[l]{shouldの書き換え}
		\answer{8}{ought to, had better}
	\end{itembox}
	\begin{itembox}[l]{used toの書き換え}
		\answer{8}{would often(よく〜したものだ)}
	\end{itembox}
	\begin{itembox}[l]{mustの否定}
		\answer{8}{推量ならcan't, 不必要don't have to}
	\end{itembox}

\end{multicols}

\question{
	\begin{itembox}[l]{過去の表現方法の違い}
		\vspace{2cm}
	\end{itembox}
}{}

\page{}
\question{\section{助動詞 日本語$\rightarrow$英語}}{\stepcounter{section}}


\begin{itembox}[l]{単語}
	\begin{multicols}{2}
		\begin{itemize}
			\item 〜できる \answer{5}{can}
			\item 〜かもしれない \answer{5}{may}
			\item 〜はずである \answer{5}{should}
			\item 〜の可能性がある \answer{5}{can}
			\item 〜しても良い \answer{5}{may}
			\item 〜しなければならない \answer{5}{must}
			\item かつて〜だった \answer{5}{used to}
			\item よく〜したものだ(2) \answer{5}{would often, used to}
			\item 〜する必要がある \answer{5}{have to}
			\item 〜すべきである(2) \answer{5}{should, had better}
			\item 〜しませんか \answer{5}{Shall we?}
			\item 〜に違いない \answer{5}{must}

			\item 〜でしょう \answer{5}{will}
			\item 〜するつもりある \answer{5}{be going to}
			\item 〜しましょうか \answer{5}{Shall I?}
			\item 〜した方がよい \answer{5}{had better}
		\end{itemize}
	\end{multicols}
\end{itembox}


\begin{multicols}{2}
	\begin{itembox}[l]{be going toの書き換え}
		\answer{8}{will}
	\end{itembox}
	\begin{itembox}[l]{have toの書き換え}
		\answer{8}{must}
	\end{itembox}
	\begin{itembox}[l]{be able toの書き換え}
		\answer{8}{can}
	\end{itembox}
	\begin{itembox}[l]{ought to/had betterの書き換え}
		\answer{8}{should}
	\end{itembox}
	\begin{itembox}[l]{can'tの否定}
		\answer{8}{must}
	\end{itembox}
\end{multicols}



\begin{itembox}[l]{過去の表現方法の違い}
	\answer{20}{
		\begin{itemize}
			\item 動作に対する時は単純に助動詞を過去形にする。
			\item 推量の意味の時は完了形を使って過去を表す。
		\end{itemize}
	}
\end{itembox}

\newpage

\section{不定詞 動名詞}

\begin{itembox}[l]{不定詞と動名詞の意味と使い方}
	\answer{8}{
		\begin{description}
			\item[不定詞] to do、〜すること・〜すべき・〜ために、not to do
			\item[動名詞] doing、〜すること、not doing
		\end{description}
		*イメージとして動名詞が過去・不定詞が未来を表す違いがある
	}
\end{itembox}

\begin{itembox}[l]{目的語について}
	\begin{itemize}
		\item 不定詞と動名詞の両方を目的語にとれる動詞  \answer{10}{\\begin / start / continue / like / love / hate}
		\item 不定詞のみを目的語にとれる動詞  \answer{10}{\\decide / expect / hope / promise / refuse / wish}
		\item 動名詞のみを目的語にとれる動詞  \answer{10}{\\admit(認める) / avoid(叫ぶ) / consider(考える) / deny(否定する) /  enjoy / finish / mind(いやがる) / miss / stop / give up / put off}
		\item 目的語が動名詞か不定詞で意味が違う  \answer{10}{
			      \begin{itemize}
				      \item forget doing 〜したことを忘れません/ forget to do 〜するのを忘れないように
				      \item remember doing 〜したことを覚えておく / remember to do 〜するのを覚えておく
				      \item try doing 〜してみる / try to do 〜しようとする
				      \item regret doing 〜したことを後悔する / regret to do 残念ながら〜する
			      \end{itemize}
		      }
	\end{itemize}
\end{itembox}

\begin{itembox}[l]{疑問詞+to do(表現5、意味、書き換え)}
	\answer{10}{
		\begin{multicols}{2}
			\begin{itemize}
				\item what to do 何をすべきか
				\item when to do いつすべきか
				\item where to do どこですべきか
				\item which (名詞) to do どれをすべきか
				\item how to do どのようにすべきか、〜のやり方
			\end{itemize}
		\end{multicols}
		I don't know what to do. = I don't know what I should do.
	}
\end{itembox}

\begin{itembox}[l]{S V O to do(5)}
	\answer{10}{
		\begin{itemize}
			\item want+O+to(O に〜してもらいたい)
			\item expect+O+to(O が〜するだろうと思う[期待する])
			\item tell/ask/advice+O+to
			      (O に〜するよう言う/頼む/助言する)
		\end{itemize}
	}
\end{itembox}

\begin{itembox}[l]{原形不定詞2、動詞の例}
	\answer{10}{
		V 人 do 人がdoなのをV
		\begin{description}
			\item[知覚動詞] see / watch / look at / hear / listen to / feel / notice(気づく)
			\item[使役動詞] make(無理やりさせる) / have(〜させる、〜してもらう) / let(〜させる、〜させてやる)
		\end{description}
	}
\end{itembox}
\begin{itembox}[l]{be 不定詞(意味3)}
	\answer{10}{
		\begin{itemize}
			\item 〜なことになっている(予定・義務)
			\item 従うべき =should
			\item 〜ひとつ...だった =could
		\end{itemize}
	}
\end{itembox}

\begin{itembox}[l]{不定詞の慣用表現}
	\begin{multicols}{2}
		\begin{itemize}
			\item 〜することは〜にとって〜だ(2、違い)  \answer{10}{\\It be 形 for(of) 人 to do\\of:人の性質を表す形容詞 kind/good/nice,wise/brave(勇敢な),careless,foolish}
			\item 〜するには十分〜だ  \answer{10}{enough 形 to do}
			\item 〜するには〜すぎる  \answer{10}{too 形 to do}
			\item 〜するために(2)  \answer{10}{in order to, so as to}
			\item 〜したが  \answer{10}{only to do}
			\item 〜し2度と・・・  \answer{10}{never to do}
			\item 実を言うと  \answer{10}{to tell the truth}
			\item いわば  \answer{10}{so to speak}
			\item 言うまでもなく  \answer{10}{needless to say}
			\item まず第一に  \answer{10}{to begin with}
			\item 確かに  \answer{10}{to be sure}
			\item 簡潔に言えば  \answer{10}{to be brief}
			\item 奇妙なことに  \answer{10}{strange to say}
			\item 率直に言えば  \answer{10}{to be frank with you}
		\end{itemize}
	\end{multicols}
\end{itembox}

\begin{itembox}[l]{動名詞の慣用表現}
	\begin{multicols}{2}
		\begin{itemize}
			\item 〜しませんか  \answer{10}{how about ing}
			\item 〜を楽しみにしている  \answer{10}{look forward to ing}
			\item 〜するのに慣れている  \answer{10}{be used to ing}
			\item 〜する気がしない  \answer{10}{feel like ing}
			\item 〜する価値がある  \answer{10}{be worth ing}
			\item 〜するとすぐに  \answer{10}{on ing}
			\item 考えざを得ない  \answer{10}{cant't help ing}
			\item 〜しても無駄だ  \answer{10}{It is no use ing}
			\item 〜できない  \answer{10}{there is no ing}
		\end{itemize}
	\end{multicols}
\end{itembox}

\newpage

\section{分詞}

\begin{itembox}[l]{基本形(意味2、表現2、使い分け)}
	\answer{20}{
		名詞を修飾する手段\\
		現在分詞 doing 〜している、過去分詞 done 〜される\\
		修飾語の塊が1語なら名詞の前、2語以上なら名詞の後ろに置く
	}
\end{itembox}

\begin{itembox}[l]{動詞の補語となる例}
	\answer{20}{
		\begin{itemize}
			\item S が C する/C される S+V+C(現在分詞/過去分詞)
			      \begin{itemize}
				      \item keep[remain/look/seem/feel]+分詞
				      \item come[stand/sit/lie]+分詞
			      \end{itemize}

			\item O が C している/C される S+V+O+C(現在分詞/過去分詞)
			      \begin{itemize}
				      \item S+知覚動詞(see/watch/hear/feel)+O+分詞
				      \item S+使役動詞(make/have)+O+分詞
				      \item S+{keep/leave/find}+O+分詞
			      \end{itemize}

		\end{itemize}
	}
\end{itembox}

\begin{itembox}[l]{分詞構文のポイント}
	\answer{20}{

		\begin{itemize}
			\item 接続詞を省略することができる。
			\item 主語が同じなら片方省略
			\item 時制の一致(1段階のみのずれはhaveを使って辻褄を合わせる)
			\item be動詞は省略可能
		\end{itemize}
	}
\end{itembox}

\begin{itembox}[l]{分詞構文の用法(6)}
	\answer{30}{
		\begin{itemize}
			\item 〜とき
			\item 〜だから
			\item 〜して
			\item 〜しながら
			\item もし〜すれば
			\item 〜だが
		\end{itemize}
	}
\end{itembox}

\newpage
\section{関係詞}

\begin{itembox}[l]{関係代名詞(表現5、使い分け)}
	\answer{20}{
		\begin{itemize}
			\item which  ものの時に使う
			\item who 人の主格に対して使う
			\item that なんでも使えるがtheなどがついて先行詞が特定されている時にはよく使う
			\item whom 人の目的格に対して使う
			\item whose 所有格に対して使う
		\end{itemize}
	}
\end{itembox}

\begin{itembox}[l]{特別な関係代名詞(表現1、使い方、何と等価か)}
	\answer{10}{
		what = things which 〜なもの
	}
\end{itembox}

\begin{itembox}[l]{関係副詞(表現4、使い分け)}
	\answer{17}{
		先行詞が副詞として関係代名詞節に補う
		\begin{multicols}{4}
			\begin{itemize}

				\item where 場所
				\item when 時
				\item why 理由
				\item how 方法
			\end{itemize}
		\end{multicols}
	}
\end{itembox}

\begin{itembox}[l]{制限用法と非制限用法}
	\begin{itemize}
		\item I have a brother who can speak English. \answer{8}{
			      \\制限用法:兄弟は他にもいるかも知れず、そのうちの一人が英語を話せる。
		      }
		\item I have a brother, who can speak English. \answer{8}{
			      \\非制限用法:兄弟は一人で、その人が英語を話せる
		      }
	\end{itemize}
\end{itembox}

\begin{itembox}[l]{複合関係代名詞}
	\answer{10}{
		\begin{tabular}{|l|l|l|}
			\hline
			複合関係代名詞 & 意味                                 & 主な書きかえ               \\ \hline
			whoever        & $\sim$する人は誰でも                 & anyone   who$\sim$         \\ \hline
			whichever      & $\sim$するものはどれ{[}どちら{]}でも & \begin{tabular}[c]{@{}l@{}}any   one {[}ones{]} that$\sim$   , either   (one) that$\sim$\end{tabular} \\ \hline
			whatever       & $\sim$するものは何でも               & anything   that $\sim$     \\ \hline
		\end{tabular}
		\\
		\begin{tabular}{|l|l|l|}
			\hline
			複合関係代名詞 & 意味                                        & 主な書きかえ             \\ \hline
			whoever        & $\sim${[}誰が誰を{]}$\sim$しようとも        & no   matter who          \\ \hline
			whichever      & どれ{[}どちら{]}が{[}を{]} $\sim$しようとも & no   matter which $\sim$ \\ \hline
			whatever       & 何が{[}何を{]}$\sim$しようとも              & no   matter what         \\ \hline
		\end{tabular}
	}
\end{itembox}

\begin{itembox}[l]{複合関係副詞}
	\answer{10}{
		\begin{tabular}{|l|l|l|}
			\hline
			複合関係副詞 & 意味                                 & 主な書きかえ            \\ \hline
			whenever     & $\sim$する時はいつでも               & any   time              \\ \hline
			wherever     & $\sim$するところはどこへ{[}で{]}でも & (at)   any place $\sim$ \\ \hline
			whatever     & 何が{[}何を{]}$\sim$しようとも       & no   matter what        \\ \hline
		\end{tabular}
		\\
		\begin{tabular}{|l|l|l|}
			\hline
			複合関係副詞             & 意味                           & 主な書きかえ                            \\ \hline
			whenever                 & いつ$\sim$しようとも           & no   matter when                        \\ \hline
			wherever                 & どこへ{[}で{]}$\sim$しようとも & no   matter where                       \\ \hline
			however+形容詞{[}副詞{]} & どんなに$\sim$でも             & no   matter how+形容詞{[}副詞{]} $\sim$ \\ \hline
		\end{tabular}
	}
\end{itembox}



\begin{itembox}[l]{重要表現}
	\begin{multicols}{2}
		\begin{itemize}
			\item いわゆる  \answer{5}{what is called}
			\item さらにいいことには  \answer{5}{what is better}
			\item さらに悪いことには  \answer{5}{what is worse}
			\item さらに  \answer{5}{what is more}
			\item 今(昔)の〜  \answer{5}{what S be}
			\item AとBの関係はCとDの関係に等しい \\ \answer{5}{A is to what C is to D}
		\end{itemize}
	\end{multicols}
\end{itembox}


\newpage
\section{比較}

\begin{itembox}[l]{比較級(意味、表現、よく使う前置詞とその意味)}
	\answer{10}{〜より〜だ、形容詞er/more 形容詞、than(〜より)}
\end{itembox}

\begin{itembox}[l]{最上級(意味、表現、よく使う前置詞とその意味)}
	\answer{20}{一番〜だ、the 形容詞est/most 形容詞、in 集団/of 数字(〜のなかで)}
\end{itembox}

\begin{itembox}[l]{比較級・最上級の不規則変化、good/well/many/much/bad/little/few}
	\begin{multicols}{2}
		\begin{itemize}
			\item good/well  \answer{5}{better best}
			\item many/much  \answer{5}{more most}
			\item bad  \answer{5}{worse worst}
			\item little/few  \answer{5}{less least}
		\end{itemize}
	\end{multicols}
\end{itembox}



\begin{itembox}[l]{同等比較(意味、表現)}
	\answer{10}{as 原級 as、〜と同じくらい〜}
\end{itembox}

\begin{itembox}[l]{比較級と最上級のそれぞれの強調}
	\answer{10}{
		ずば抜けて〜
		\begin{multicols}{3}
			\begin{itemize}
				\item 両方 much
				\item 比較級 far
				\item 最上級 by far
			\end{itemize}
		\end{multicols}
	}
\end{itembox}

\begin{itembox}[l]{比較の差を表す前置詞}
	\answer{10}{by (older than brother by two years)}
\end{itembox}

\begin{itembox}[l]{慣用表現}
	\begin{multicols}{2}
		\begin{itemize}
			\item 〜のX倍  \answer{5}{X times as 原級 as}
			\item だんだん〜  \answer{5}{比較級 and 比較級}
			\item できる限り(2)  \answer{5}{\\as 原級 as possible, as 原級 as one can}
			\item 〜すればするほど〜だ  \answer{5}{the 比較級, the 比較級}
			\item どの〜よりも  \answer{5}{than any other 名詞}
			\item 〜ほど〜なのはない  \answer{5}{No one can as 原級 as ~}
			\item 〜というよりはむしろ〜  \answer{5}{not so much A as B}
			\item 二番目に〜  \answer{5}{the secound 最上級}
			\item 〜するほど馬鹿ではない \answer{5}{know better than}
			\item もはや〜ない \answer{5}{no longer}
		\end{itemize}
	\end{multicols}
\end{itembox}

\begin{itembox}[l]{thanを使わない形容詞}
	\answer{10}{
		\begin{itemize}
			\item superior/inferior to〜(〜より優れて/劣って)
			\item senior/junior to〜(〜より地位が上/下)
			\item prefer A to B(BよりAを好む)
		\end{itemize}
	}
\end{itembox}


\begin{itembox}[l]{比較級・最上級が2種類}
	\begin{tabular}{|llll|l|l|}
		\hline
		\multicolumn{4}{|l|}{原級} & 比較級                                          & 最上級                                                                                                                                 \\ \hline
		\multicolumn{1}{|l|}{far}  & \multicolumn{1}{l|}{\begin{tabular}[c]{@{}l@{}}形\question{     }{遠い}\\ 形\question{   }{それ以上の }\end{tabular}} & \multicolumn{1}{l|}{\begin{tabular}[c]{@{}l@{}}副\question{     }{遠く}\\ 副\answer{0}{さらに}\end{tabular}} & \begin{tabular}[c]{@{}l@{}}{[}距離{]}\\ {[}程度{]}\end{tabular} & \begin{tabular}[c]{@{}l@{}}\question{     }{farther}\\\question{     }{ further}\end{tabular} & \begin{tabular}[c]{@{}l@{}}\question{     }{farthest}\\ \answer{0}{furthest}\end{tabular} \\ \hline
		\multicolumn{1}{|l|}{late} & \multicolumn{1}{l|}{\begin{tabular}[b]{@{}l@{}}形\answer{0}{遅い}\\ 形\answer{0}{後の}\end{tabular}} & \multicolumn{1}{l|}{\begin{tabular}[c]{@{}l@{}}副\answer{0}{遅く}\\  \\ 副\answer{0}{後で}\end{tabular}} & \begin{tabular}[c]{@{}l@{}}{[}時間{]}\\ {[}順序{]}\\ {[}順序{]}\end{tabular} & \begin{tabular}[b]{@{}l@{}}\answer{0}{later}\\ \answer{0}{latter}\end{tabular} & \begin{tabular}[c]{@{}l@{}}\answer{0}{latest}\\ \answer{0}{last}\\ \answer{0}{last}\end{tabular} \\ \hline
		\multicolumn{1}{|l|}{old}  & \multicolumn{3}{l|}{\begin{tabular}[c]{@{}l@{}}形\answer{0}{年をとった、古い}\\ 形\answer{0}{(兄弟のうちで)年長の}\end{tabular}} & \begin{tabular}[c]{@{}l@{}}\answer{0}{older}\\ \answer{0}{elder}\end{tabular}                      & \begin{tabular}[c]{@{}l@{}}\answer{0}{oldest}\\ \answer{0}{eldest}\end{tabular}                                                           \\ \hline
	\end{tabular}
\end{itembox}

\newpage


\section{話法}
\begin{itembox}[l]{基本}
	\begin{description}
		\item[直接話法] \answer{8}{人が言ったことをそのまま英文にする。""で囲まれたもの。S say to 人 "文言"}
		\item[間接話法] \answer{8}{人が〜を言っていたよみたいな感じ、言ったことそのままとは限らない。""がなく時制の一致が行われる。代名詞も変わることがある。}
			\begin{description}
				\item[普通の文] \answer{5}{say that, tell 人 that}
				\item[疑問文] \answer{5}{ask 人 間接疑問文}
				\item[命令文] \answer{5}{tell 人 (not) to do}
				\item[Pleaseの命令文] \answer{5}{ask 人 to do}
				\item[Let'sの命令文] \answer{5}{suggest 人 that S (should) V}
				\item[and, or, but] \answer{5}{接続詞 that}
				\item[because, so]  \answer{5}{that いらない}
			\end{description}
	\end{description}
	\answer{0}{
		*現在のことでなくても現在で書くような文については時制の一致を受けない。}
\end{itembox}

\begin{itembox}[l]{時制の一致による変化}
	\begin{multicols}{2}
		\begin{itemize}
			\item 過去,"現在" \answer{5}{過去,過去}
			\item 過去,"過去" \answer{5}{過去,過去完了}
			\item here \answer{5}{there}
			\item yesterday \answer{5}{the day before}
			\item now \answer{5}{then}
			\item today \answer{5}{that day}
			\item tommorow \answer{5}{the next day(the following day)}
			\item last night \answer{5}{the night before(the previous night)}
			\item next week \answer{5}{the next week}
			\item ago \answer{5}{before}
			\item this \answer{5}{that}
			\item these \answer{5}{those}
		\end{itemize}
	\end{multicols}
\end{itembox}


\begin{itembox}[l]{間接疑問文(疑問詞がある場合とない場合)}
	\begin{description}
		\item[疑問詞がない] \answer{5}{if(whether) S V}
		\item[疑問詞がある] \answer{5}{疑問詞 S V または 疑問詞 V}
	\end{description}
\end{itembox}

\newpage

\section{仮定法}
\begin{itembox}[l]{仮定法、if文との違い}
	\answer{10}{ありえないことを言うのが仮定法、時制を一つずらす\\
		現在のことは過去、過去のことは過去完了でかく}
\end{itembox}

\begin{itembox}[l]{仮定法未来}
	\begin{itemize}
		\item 万が一SがVするならば \answer{5}{If S should V}
		\item 仮にSがVするならば \answer{5}{If S were to V}
	\end{itemize}
\end{itembox}

\begin{itembox}[l]{仮定法現在}
	\answer{0}{後ろに続くthat説にはshouldが来る。shouldは省略されることがほとんど}
	\begin{itemize}
		\item 動詞(7) \answer{5}{demand, suggest, advise, insist, recommend, require, request}
		\item 形容詞(3) \answer{5}{important, necessary, essential}
	\end{itemize}
\end{itembox}

\begin{itembox}[l]{ifの省略の語順}
	\answer{10}{If S V = V S}
\end{itembox}

\begin{itembox}[l]{重要表現}
	\begin{multicols}{2}
		\begin{itemize}
			\item Would it be possible to V? \answer{5}{\\〜することは可能でしょうか}
			\item Would you mind if S V(過去)? \answer{5}{\\〜してもいいですか}
			\item Could you V? \answer{5}{〜してもらえますか}
			\item Could I V? \answer{5}{〜してもいいですか}
			\item I wonder if S V(過去)\answer{5}{〜かしらと思う}
			\item I was wondering if S V(過去) \answer{5}{〜かしらと思う}
			\item To hear S speak O \answer{5}{もしSがOを話すのを聞けば、}
		\end{itemize}
	\end{multicols}
	\begin{multicols}{2}
		\begin{itemize}
			\item まるで〜 \answer{5}{as if(though)}
			\item もう〜する時間だ \answer{5}{It is time S V\\}
			\item 〜にもかかわらず \answer{5}{otherwise}
			\item もし〜がなければ(3) \answer{8}{\\If it were not for, Without, But for}
			\item 〜さえすればなあ \answer{5}{If only, wishとほぼ同じ}
		\end{itemize}
	\end{multicols}
\end{itembox}

\newpage


\subsubsection*{前置詞}

{\renewcommand\arraystretch{
		\ifanswer
			1.0
		\else
			1.8
		\fi}
	\begin{table}[H]
		\centering
		\begin{tabular}{|c|p{2cm}||c|p{2cm}||c|p{2cm}|}
			\hline
			意味                 & 単語                & 意味           & 単語                & 意味             & 単語                \\ \hline\hline
			〜の上に             & \answer{0}{on}      & 〜で、〜に     & \answer{0}{at}      & 〜の間に(時間) & \answer{0}{for}     \\ \hline
			〜の下に             & \answer{0}{under}   & 〜といっしょに & \answer{0}{with}    & 〜の間に(時間) & \answer{0}{during}  \\ \hline
			〜の中に             & \answer{0}{in}      & 〜の           & \answer{0}{of}      & 〜の間に(場所) & \answer{0}{between} \\ \hline
			〜の中へ             & \answer{0}{into}    & 〜のために     & \answer{0}{for}     & 〜の後に         & \answer{0}{after}   \\ \hline
			〜の近くに           & \answer{0}{near}    & 〜によって     & \answer{0}{by}      & 〜の前に         & \answer{0}{before}  \\ \hline
			〜のそばに           & \answer{0}{by}      & 〜のように     & \answer{0}{like}    & 〜について       & \answer{0}{about}   \\ \hline
			〜から               & \answer{0}{from}    & 〜にとって     & \answer{0}{for}     & 〜まで           & \answer{0}{until}   \\ \hline
			〜へ                 & \answer{0}{to}      & 〜なしで       & \answer{0}{without} & 〜までに         & \answer{0}{by}      \\ \hline

			〜以内に             & \answer{0}{within}  & 〜後に         & \answer{0}{in}      & 〜として         & \answer{0}{as}      \\ \hline
			〜に反対して         & \answer{0}{against} & 〜賛成して     & \answer{0}{for}     & 〜を通して       & \answer{0}{through} \\ \hline
			〜の間に(三つ以上) & \answer{0}{among}   & 〜の上方に     & \answer{0}{over}    & 〜を横切って     & \answer{0}{across}  \\ \hline
			〜以来               & \answer{0}{since}   &                &                     &                  &                     \\ \hline
		\end{tabular}
	\end{table}
}

\subsubsection*{接続詞}


{\renewcommand\arraystretch{\ifanswer
			1.0
		\else
			1.8
		\fi}
	\begin{table}[H]
		\centering
		\begin{tabular}{|c|p{3cm}||c|p{3cm}||c|p{3cm}|}
			\hline
			           & 意味                     &                 & 意味                                                         &         & 意味                           \\ \hline\hline
			unless     & \answer{0}{〜しない限り} & though/although & \answer{0}{〜だが}                                           & in case & \answer{0}{〜の場合に備えて}   \\\hline
			so         & \answer{0}{だから}       & as              & \answer{0}{〜するにつれて、〜するとき、〜なので、〜のように} & since   & \answer{0}{〜なので、〜以来}   \\\hline
			even if    & \answer{0}{たとえ〜でも} & as soon as      & \answer{0}{〜するとすぐに}                                   & while   & \answer{0}{〜の間、一方で}     \\\hline
			if/whether & \answer{0}{〜かどうか}   & that            & \answer{0}{〜ということ、同格}                               & for     & \answer{0}{というのも〜なので} \\\hline
			%& \answer{0}{}&&\answer{0}{}&&\answer{0}{}\\\hline
		\end{tabular}
	\end{table}
}

\begin{itembox}[l]{接続詞と前置詞の違い}
	\answer{10}{
		\begin{description}
			\item[接続詞] 後ろにS V
			\item[前置詞] 後ろに名詞
		\end{description}
	}
\end{itembox}

\end{document}