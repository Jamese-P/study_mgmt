\newif\ifanswer\answertrue
% \answerfalse

\documentclass[10pt,dvipdfmx]{jsarticle}
\usepackage[margin=15truemm]{geometry}
\usepackage[dvipdfmx]{graphicx}
\usepackage{wrapfig}
\usepackage[dvipdfmx]{color}
\usepackage{ascmac} % for scree
\usepackage{subfigure} % for subfigure
\usepackage{multicol}
\usepackage{setspace}
\usepackage{diagbox}
\usepackage{fancyhdr}
\usepackage{xcolor}
\usepackage{here}
\usepackage{amsmath}
\usepackage{amssymb}
\usepackage{tikz}
\usetikzlibrary{intersections, calc, arrows.meta}
\pagestyle{fancy}
\ifanswer
\lhead{数I 解答}
\else
\lhead{数I}
\fi
\rhead{\number\month\number\day}

\ifanswer
\newcommand{\answer}[2]{{\color{orange}#2}}
\newcommand{\page}[2]{#1}
\newcommand{\question}[2]{{\color{orange}#2}}
\else
\newcommand{\answer}[2]{\vspace{#1mm}}
\newcommand{\page}[2]{#2}
\newcommand{\question}[2]{#1}
\fi%answer

\begin{document}
\subsection*{数と式}
\subsubsection*{展開}
\begin{Large}
  \begin{itemize}
    \item $(a+b+c)^2=$
    \item $(a+b)^3=$
    \item $(a-b)^3=$
    \item $(x+y)(x^2-xy+y^2)=$
    \item $(x-y)(x^2+xy+y^2)=$
  \end{itemize}
\end{Large}

\subsubsection*{因数分解}
\begin{Large}
  \begin{itemize}
    \item $a^2+b^2+c^2+2ab+2bc+2ca=$
    \item $x^3+3x^2y+3xy^2+y^3=$
    \item  $x^3-3x^2y+3xy^2-y^3=$
    \item $x^3+y^3=$
    \item $x^3-y^3=$
    \item $x^3+y^3+z^3-3xyz=$
  \end{itemize}
\end{Large}
\begin{itembox}[l]{因数分解の手順}
  \begin{Large}
    \begin{enumerate}
      \item \answer{0}{降べきの順に並べる}
      \item \answer{0}{共通因数をくくる}%共通因数
      \item \answer{0}{公式} %公式
      \item \answer{0}{襷掛け} %襷掛け
    \end{enumerate}
  \end{Large}
\end{itembox}

\begin{itembox}[l]{例題}
  \begin{large}
    \begin{enumerate}
      \item $3x^2+10x+3=$\answer{0}{$(3x+1)(x+3)$}
      \item $x^2+xy-2y^2+4x+17y-21=$\answer{0}{$(x+2y-3)(x-y+7)$}
      \item $a^2b+ab^2+b^2c+bc^2+c^2a+ca^2+2abc=$\answer{0}{$(a+b)(b+c)(c+a)$}
    \end{enumerate}
  \end{large}
\end{itembox}

\subsubsection*{絶対値}
\begin{itembox}[l]{例題}
  \begin{large}
    \begin{enumerate}
      \item $|\pi-4|=$\answer{0}{$4-\pi$}
      \item $|\sqrt{2}-1|+|\sqrt{2}-3|=$\answer{0}{2}
    \end{enumerate}
  \end{large}
\end{itembox}
\subsection*{分母の有利化}
\begin{itembox}[l]{例題}
  \begin{large}
    \begin{enumerate}
      \item $\cfrac{1}{\sqrt{5}-\sqrt{3}}=$\answer{0}{$\cfrac{\sqrt{5}+\sqrt{3}}{2}$}
    \end{enumerate}
  \end{large}
\end{itembox}
\subsubsection*{二重根号}
\begin{Large}
  $\sqrt{(x+a)^2}=$
\end{Large}
\begin{itembox}[l]{例題}
  \begin{large}
    \begin{enumerate}
      \item $\sqrt{6-\sqrt{20}}=$\answer{0}{$\sqrt{5}-1$}
      \item $\sqrt{14-4\sqrt{10}}=$\answer{0}{$\sqrt{10}-2$}
      \item $\sqrt{2+\sqrt{3}}=$\answer{0}{$\cfrac{\sqrt{6}+\sqrt{2}}{2}$}
    \end{enumerate}
  \end{large}
\end{itembox}

\subsubsection*{対象式}
\begin{itembox}[l]{例題}
  $a=\cfrac{\sqrt{3}}{\sqrt{2}+1}, b=\cfrac{\sqrt{3}}{\sqrt{2}-1}$
  \begin{large}
    \begin{enumerate}
      \item $a+b$ \answer{0}{$2\sqrt{6}$}
      \item $ab$ \answer{0}{3}
      \item $a^2+b^2$ \answer{0}{18}
      \item $a^3+b^3$ \answer{0}{$30\sqrt{6}$}
    \end{enumerate}
  \end{large}
\end{itembox}



\subsubsection*{一次不等式}
\begin{itembox}[l]{ポイント}
  \answer{8}{負の数で割るときに不等号を逆向きにする。それ以外は普通の方程式}

\end{itembox}
\begin{itembox}[l]{例題}
  \begin{large}
    \begin{enumerate}
      \item $x-5>3(7x-5)$ \answer{0}{$x<\cfrac{1}{2}$}
      \item $\frac{x+1}{2}\leqq\frac{2x+4}{3}$ \answer{0}{$x\geqq-5$}
    \end{enumerate}
  \end{large}
\end{itembox}

\subsubsection*{絶対値を含む等式・不等式}
\begin{itembox}[l]{ポイント}
  \begin{Large}
    \begin{itemize}
      \item \answer{0}{絶対値の外に文字がない時は$\pm$}
      \item \answer{0}{絶対値の外に文字がある時は場合分け}
    \end{itemize}
  \end{Large}
\end{itembox}

\begin{itembox}[l]{例題}
  \begin{large}
    \begin{enumerate}
      \item $|5-x|=2$  \answer{0}{$x=3, 7$}
      \item $|x-2|=2x-7$  \answer{0}{$x=5$}
      \item $|x-5|<3$  \answer{0}{$2<x<8$}
      \item $|x-5|\geqq3$  \answer{0}{$x\leqq2, 8\leqq x$}
      \item $|2x-3|\geqq5x+1$  \answer{0}{$x=\cfrac{2}{7}$}
      \item $|x-2|+|x+1|=x+3$  \answer{0}{$x=0, 4$}
    \end{enumerate}
  \end{large}
\end{itembox}

\newpage
\subsection*{二次関数}
\subsubsection*{一般式(2) グラフをかけ}
\begin{large}
  \begin{itemize}
    \item \answer{0}{平方完成のパターン、頂点や軸がわかる}
    \item \answer{0}{展開のパターン、y切片がわかる}
  \end{itemize}
\end{large}

\begin{itembox}[l]{ポイント}
  \answer{8}{必ずグラフを書く}
\end{itembox}



\subsubsection*{文字を含む最大最小}
\begin{itembox}[l]{場合分けの仕方(下に凸の場合)}
  \answer{0}{軸がどの位置にあるかで場合わけを行う\\}
  最小値
  \answer{50}{
    軸が範囲の外か中か
  }

  最大値
  \answer{50}{軸が範囲の中央か左右か}


\end{itembox}

\subsubsection*{解の個数の調べ方}
\begin{large}
  \begin{itemize}
    \item
  \end{itemize}
\end{large}

\subsubsection*{解の種類}
$f(x)=ax^2+bx+c=0$の解
\begin{multicols}{3}
  \begin{large}
    \begin{itemize}
      \item 二つの正の解
            \begin{itemize}
              \item \item \item
            \end{itemize}
      \item 二つの負の解
            \begin{itemize}
              \item \item \item
            \end{itemize}
      \item 正の解と負の解
            \begin{itemize}
              \item \item
            \end{itemize}
    \end{itemize}
  \end{large}
\end{multicols}

\subsubsection*{二次不等式}
\begin{itembox}[l]{例題}
  \begin{large}
    \begin{multicols}{2}
      \begin{enumerate}
        \item $x^2-4x+3>0$  \answer{10}{$x<1, 3<x$}
        \item $x^2-4x+3\leqq0$ \answer{10}{$1\leqq x\leqq 3$}
        \item $x^2-4x+7\leqq0$ \answer{10}{$解なし$}
        \item $x^2-4x+4\geqq0$ \answer{10}{$全ての実数$}
        \item $x^2-4x+4>0$ \answer{10}{$x\not=2$}
        \item  $x^2-4x+4<0$ \answer{10}{$解なし$}
        \item $x^2-4x+4\leqq0$ \answer{10}{$x=2$}
      \end{enumerate}
    \end{multicols}
  \end{large}

\end{itembox}

\subsubsection*{解と係数の関係}
$ax^2+bx+c=0$の解を$\alpha,\beta$とする
\begin{multicols}{2}
  \begin{itemize}
    \item \item
  \end{itemize}
\end{multicols}

\begin{multicols}{3}
  \begin{large}
    \begin{itemize}
      \item 二つの正の解
            \begin{itemize}
              \item \item \item
            \end{itemize}
      \item 二つの負の解
            \begin{itemize}
              \item \item \item
            \end{itemize}
      \item 正の解と負の解
            \begin{itemize}
              \item \item
            \end{itemize}
    \end{itemize}
  \end{large}
\end{multicols}


\newpage
\subsection*{図形}
\begin{table}[h]
  \begin{minipage}[h]{0.3\textwidth}
    \subsubsection*{定義}
    \begin{tikzpicture}
      \draw(0,0)--(4,0)--(4,3)--cycle;
    \end{tikzpicture}
  \end{minipage}
  \begin{minipage}[h]{0.7\textwidth}
    \subsubsection*{代表角}
    {\renewcommand\arraystretch{2}
      \begin{table}[H]
        \begin{tabular}{|c||p{1cm}|p{1cm}|p{1cm}|p{1cm}|p{1cm}|p{1cm}|p{1cm}|}
          \hline
          代表角 &  &  &  &  &  &  & \\
          \hline
          sin    &  &  &  &  &  &  & \\
          \hline
          cos    &  &  &  &  &  &  & \\
          \hline
          tan    &  &  &  &  &  &  & \\
          \hline
        \end{tabular}
      \end{table}
    }
  \end{minipage}

\end{table}




\subsubsection*{相互関係の公式}
\begin{multicols}{3}
  \begin{Large}
    \begin{itemize}
      \item
      \item
      \item
    \end{itemize}
  \end{Large}
\end{multicols}


\subsubsection*{補角}
\begin{multicols}{4}
  \begin{itemize}
    \item $180-\theta$
          \begin{itemize}
            \item $\sin(180-\theta)$
            \item $\cos(180-\theta)$
            \item $\tan(180-\theta)$
          \end{itemize}
    \item $180+\theta$
          \begin{itemize}
            \item $\sin(180+\theta)$
            \item $\cos(180+\theta)$
            \item $\tan(180+\theta)$
          \end{itemize}
    \item $90-\theta$
          \begin{itemize}
            \item $\sin(90-\theta)$
            \item $\cos(90-\theta)$
            \item $\tan(90-\theta)$
          \end{itemize}
    \item $90+\theta$
          \begin{itemize}
            \item $\sin(90+\theta)$
            \item $\cos(90+\theta)$
            \item $\tan(90+\theta)$
          \end{itemize}
  \end{itemize}

\end{multicols}

\subsubsection*{正弦定理}
\begin{Large}
  \begin{itemize}
    \item
  \end{itemize}
\end{Large}
\subsubsection*{余弦定理}
\begin{Large}
  \begin{itemize}
    \item  \item  \item
  \end{itemize}
\end{Large}

\begin{itembox}[l]{正弦定理と余弦定理の使い分け}
  \answer{8}{角度が二箇所わかっていれば正弦定理、全ての辺か二つの辺と角がわかっているなら余弦定理}
\end{itembox}


\subsubsection*{面積の求め方}
\answer{0}{sinを使うのと内接円}
\begin{multicols}{2}
  \begin{Large}
    \begin{itemize}
      \item  \item
    \end{itemize}
  \end{Large}
\end{multicols}

\newpage

\subsection*{データ}
\subsubsection*{用語}
\begin{Large}
  \begin{itemize}
    \item 中央値
    \item 最頻値
    \item 範囲
    \item 四分位数
    \item 四分位範囲
    \item 四分位偏差
    \item 箱ひげ図
    \item 階級
    \item 階級値
    \item 度数
    \item 相対度数
    \item ヒストグラム
    \item 相関
  \end{itemize}
\end{Large}

\subsubsection*{分散}
\begin{Large}
  \begin{itemize}
    \item \item
  \end{itemize}
\end{Large}

\subsubsection*{標準偏差}
\begin{Large}
  \begin{itemize}
    \item
  \end{itemize}
\end{Large}

\subsubsection*{相関係数}
\begin{Large}
  \begin{itemize}
    \item
  \end{itemize}
\end{Large}
\end{document}