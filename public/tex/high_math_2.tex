\newif\ifanswer\answertrue
%\answerfalse

\documentclass[10pt,dvipdfmx]{jsarticle}
\usepackage[margin=15truemm]{geometry}
\usepackage[dvipdfmx]{graphicx}
\usepackage{wrapfig}
\usepackage[dvipdfmx]{color}
\usepackage{ascmac} % for scree
\usepackage{subfigure} % for subfigure
\usepackage{multicol}
\usepackage{setspace}
\usepackage{diagbox}
\usepackage{fancyhdr}
\usepackage{xcolor}
\usepackage{here}
\usepackage{amsmath}
\usepackage{amssymb}
\usepackage{tikz}
\usetikzlibrary{intersections, calc, arrows.meta}
\pagestyle{fancy}
\ifanswer
\lhead{数2 解答}
\else
\lhead{数2}
\fi
\rhead{\number\month\number\day}

\ifanswer
\newcommand{\answer}[2]{{\color{orange}#2}}
\newcommand{\page}[2]{#1}
\newcommand{\question}[2]{{\color{orange}#2}}
\else
\newcommand{\answer}[2]{\vspace{#1mm}}
\newcommand{\page}[2]{#2}
\newcommand{\question}[2]{#1}
\fi%answer

\begin{document}
\section*{2}
\subsection*{数と式}
\subsubsection*{二項定理}
$(a+b)^n$を展開したときの項$a^pb^q(p+q=n)$の係数
\begin{itembox}[l]{例題}
  \begin{large}
    \begin{multicols}{2}
      \begin{enumerate}
        \item $(3x-2y)^5 [x^2y^3]$
        \item $(x^2-3y)^6 [x^8y^2]$\\
        \item $(x+2y-3z)^5$
              \begin{enumerate}
                \item $[x^2yz^2]$
                \item $[xyz^3]$
              \end{enumerate}
      \end{enumerate}
    \end{multicols}

  \end{large}
\end{itembox}

\subsubsection*{恒等式}
\begin{itembox}[l]{考え方}
  \vspace{8mm}
\end{itembox}
\begin{itembox}[l]{例題}
  \begin{large}
    \begin{enumerate}
      \item $x^2+ax-5=(x-1)(x+b)$
      \item $x^3=(x-1)^3+a(x+1)^2+bx+c$
      \item $\frac{a}{x^2-1}=\frac{b}{x+1}-\frac{3}{x-1}$
    \end{enumerate}
  \end{large}
\end{itembox}

\subsubsection*{不等式の証明}
\begin{itembox}[l]{ポイント}

\end{itembox}
\begin{itembox}[l]{例題}
  \begin{large}
    $a\geqq0, b\geqq0 のとき 5\sqrt{a}+3\sqrt{b}\geqq\sqrt{25a+9b}$
    \vspace{8mm}
  \end{large}
\end{itembox}

\subsubsection*{相加相乗平均}
定義\vspace{10mm}
\begin{itembox}[l]{例題}
  \begin{enumerate}
    \item $a+\frac{4}{a}\geqq4$\vspace{10mm}
    \item $(a+\frac{1}{b})+(b+\frac{4}{a})\geqq9$\vspace{10mm}
  \end{enumerate}
\end{itembox}


\newpage
\subsection*{複素数と方程式}
\subsubsection*{基本}
\begin{large}
  \begin{itemize}
    \item 虚数単位$i$
    \item 純虚数
    \item 共役な複素数($3+i$)
  \end{itemize}
\end{large}

\subsubsection*{複素数範囲での解の種類}
$ax^2+bx+c=0$の判別式$D=$
\begin{multicols}{3}
  \begin{large}
    \begin{itemize}
      \item \item \item
    \end{itemize}
  \end{large}
\end{multicols}

\subsubsection*{二次方程式の解と係数の関係}
定義
$ax^2+bx+c=0$の解を$\alpha,\beta$とする
\begin{multicols}{2}
  \begin{itemize}
    \item \item
  \end{itemize}
\end{multicols}

解の種類
\begin{multicols}{3}
  \begin{large}
    \begin{itemize}
      \item 二つの正の解
            \begin{itemize}
              \item \item \item
            \end{itemize}
      \item 二つの負の解
            \begin{itemize}
              \item \item \item
            \end{itemize}
      \item 正の解と負の解
            \begin{itemize}
              \item \item
            \end{itemize}
    \end{itemize}
  \end{large}
\end{multicols}

\subsubsection*{高次方程式}
\begin{itembox}[l]{次数の高い方程式の因数分解}
  \begin{itemize}
    \item 因数定理で解となる候補を探す。このとき候補は\begin{Large}$\pm\frac{\hspace{3cm}}{\hspace{3cm}}$\end{Large}
    \item 組立除法
  \end{itemize}
\end{itembox}
\begin{itembox}[l]{例題}
  \begin{large}
    $x^3-3x^2-8x-4=0$
  \end{large}
  \vspace{2cm}
\end{itembox}

\newpage
\subsection*{図形と方程式}
\subsubsection*{内分と外分}
A(a)とB(b)を$m:n$
\begin{multicols}{2}
  \begin{itemize}
    \item 内分
    \item 外分
  \end{itemize}
\end{multicols}

\subsubsection*{重心}
$(x_1,y_1), (x_2,y_2), (x_3,y_3)$の重心
\vspace{10mm}

\subsubsection*{対称な点}
\begin{itembox}[l]{例題}
  \begin{large}
    \begin{enumerate}
      \item (2,3)に対して以下と対称な点
            \begin{multicols}{2}
              \begin{enumerate}
                \item $(1,-1)$
                \item $(-2,1)$
              \end{enumerate}
            \end{multicols}
      \item 直線$x-2y+7=0$に対して(1,-1)と対称な点
    \end{enumerate}
  \end{large}
\end{itembox}

\subsubsection*{直線}
\begin{itembox}[l]{例題}
  \begin{large}
    \begin{enumerate}
      \item $(-2,1)$を通る$y=-3x+9$に平行な直線
      \item $(-2,1)$を通る$y=-3x+9$に垂直な直線
    \end{enumerate}
  \end{large}
\end{itembox}

\subsubsection*{点と直線の距離}
定義 $ax+by+c=0$と$(p,q)$の距離
\vspace{10mm}

\subsubsection*{円}
一般式
\vspace{10mm}

\subsubsection*{領域}
\begin{itembox}[l]{例題}
  \begin{large}
    \begin{enumerate}
      \item $y>x^2$
      \item $y\leqq3x+1$
    \end{enumerate}
  \end{large}
\end{itembox}




\newpage
\subsection*{三角関数}
\subsubsection*{弧度法}
{\renewcommand\arraystretch{2}
  \begin{table}[H]
    \begin{tabular}{|c||p{1.5cm}|p{1.5cm}|p{1.5cm}|p{1.5cm}|p{1.5cm}|p{1.5cm}|p{1.5cm}|}
      \hline
      度数法 & 0 & 30 & 60 & 90 & 120 & 150 & 180 \\
      \hline
      弧度法 &   &    &    &    &     &     &     \\
      \hline
      sin    &   &    &    &    &     &     &     \\
      \hline
      cos    &   &    &    &    &     &     &     \\
      \hline
      tan    &   &    &    &    &     &     &     \\
      \hline
    \end{tabular}
  \end{table}
}

\begin{table}[H]
  \begin{minipage}{0.75\linewidth}
    {\renewcommand\arraystretch{2}
      \begin{table}[H]
        \begin{tabular}{|c||p{1.5cm}|p{1.5cm}|p{1.5cm}|p{1.5cm}|p{1.5cm}|p{1.5cm}|p{1.5cm}|}
          \hline
          度数法 & 210 & 240 & 270 & 300 & 330 \\
          \hline
          弧度法 &     &     &     &     &     \\
          \hline
          sin    &     &     &     &     &     \\
          \hline
          cos    &     &     &     &     &     \\
          \hline
          tan    &     &     &     &     &     \\
          \hline
        \end{tabular}
      \end{table}
    }
  \end{minipage}
  \begin{minipage}{0.2\linewidth}
    \begin{center}
      \begin{tikzpicture}
        \draw[->] (-2,0) -- (2,0) node[right] {$x$};
        \draw[->] (0,-2) -- (0,2) node[above] {$y$};
      \end{tikzpicture}
    \end{center}
  \end{minipage}
\end{table}

\subsubsection*{相互関係の公式}
\begin{multicols}{3}
  \begin{LARGE}
    \begin{itemize}
      \item
      \item
      \item
    \end{itemize}
  \end{LARGE}
\end{multicols}

\subsubsection*{三角関数の性質}
\begin{multicols}{3}
  \begin{itemize}
    \item $-\theta$
          \begin{itemize}
            \item $\sin(-\theta)$
            \item $\cos(-\theta)$
            \item $\tan(-\theta)$
          \end{itemize}
    \item $\pi-\theta$
          \begin{itemize}
            \item $\sin(\pi-\theta)$
            \item $\cos(\pi-\theta)$
            \item $\tan(\pi-\theta)$
          \end{itemize}
    \item $\pi+\theta$
          \begin{itemize}
            \item $\sin(\pi+\theta)$
            \item $\cos(\pi+\theta)$
            \item $\tan(\pi+\theta)$
          \end{itemize}
  \end{itemize}
\end{multicols}
\begin{multicols}{2}
  \begin{itemize}
    \item $\frac{\pi}{2}-\theta$
          \begin{itemize}
            \item $\sin(\frac{\pi}{2}-\theta)$
            \item $\cos(\frac{\pi}{2}-\theta)$
            \item $\tan(\frac{\pi}{2}-\theta)$
          \end{itemize}
    \item $\frac{\pi}{2}+\theta$
          \begin{itemize}
            \item $\sin(\frac{\pi}{2}+\theta)$
            \item $\cos(\frac{\pi}{2}+\theta)$
            \item $\tan(\frac{\pi}{2}+\theta)$
          \end{itemize}
  \end{itemize}

\end{multicols}

\subsubsection*{グラフ}
\begin{table}[H]
  \begin{minipage}{0.33\linewidth}
    $y=\sin\theta$
    \begin{center}
      \begin{tikzpicture}
        \draw[->] (-2,0) -- (2,0) node[right] {$x$};
        \draw[->] (0,-2) -- (0,2) node[above] {$y$};
      \end{tikzpicture}
    \end{center}
  \end{minipage}
  \begin{minipage}{0.33\linewidth}
    $y=\cos\theta$
    \begin{center}
      \begin{tikzpicture}
        \draw[->] (-2,0) -- (2,0) node[right] {$x$};
        \draw[->] (0,-2) -- (0,2) node[above] {$y$};
      \end{tikzpicture}
    \end{center}
  \end{minipage}
  \begin{minipage}{0.33\linewidth}
    $y=\tan\theta$
    \begin{center}
      \begin{tikzpicture}
        \draw[->] (-2,0) -- (2,0) node[right] {$x$};
        \draw[->] (0,-2) -- (0,2) node[above] {$y$};
      \end{tikzpicture}
    \end{center}
  \end{minipage}
\end{table}

\begin{itembox}[l]{縦幅の変化と周期の変化}
  \vspace{10mm}
\end{itembox}

\subsubsection*{加法定理}
\begin{large}
  \begin{itemize}
    \item $\sin(\alpha+\beta)=$
    \item $\sin(\alpha-\beta)=$
    \item $\cos(\alpha+\beta)=$
    \item $\cos(\alpha-\beta)=$
    \item $\tan(\alpha+\beta)=$
    \item $\tan(\alpha-\beta)=$
  \end{itemize}
\end{large}

\begin{multicols}{2}


  \subsubsection*{2倍角の公式}
  \begin{large}
    \begin{itemize}
      \item $\sin2\alpha=$
      \item $\cos2\alpha=$
      \item $\tan2\alpha=$
    \end{itemize}
  \end{large}

  \subsubsection*{半角の公式}
  \begin{large}
    \begin{itemize}
      \item $\sin\frac{\alpha}{2}=$
      \item $\cos\frac{\alpha}{2}=$
      \item $\tan\frac{\alpha}{2}=$
    \end{itemize}
  \end{large}

\end{multicols}

\subsubsection*{三角関数の合成}
$a\sin x+b\cos x$\vspace{10mm}
\begin{itembox}[l]{例題}
  \begin{large}
    \begin{enumerate}
      \item $\sin +\sqrt{3}\cos x$
      \item $\sqrt{3}\sin +\cos x$
    \end{enumerate}
  \end{large}
\end{itembox}

\newpage
\subsection*{指数関数}
\subsubsection*{基本の計算}
\begin{multicols}{3}
  \begin{Large}
    \begin{itemize}
      \item $a^0$
      \item $a^{-3}a^5$
      \item $(a^{-3})^5$
    \end{itemize}
  \end{Large}
\end{multicols}

\subsubsection*{累乗根}
\begin{multicols}{3}
  \begin{Large}
    \begin{itemize}
      \item $\sqrt[5]{32}$
      \item $\sqrt[3]{0.001}$
      \item $\sqrt[3]{-27}$
      \item $\sqrt[3]{4}\times\sqrt[3]{2}$
      \item $\sqrt[4]{243}\div\sqrt[4]{3}$
      \item $\sqrt[3]{16}+\sqrt[3]{54}$
    \end{itemize}
  \end{Large}
\end{multicols}

\subsubsection*{指数法則}
\begin{multicols}{3}
  \begin{Large}
    \begin{itemize}
      \item $(\sqrt[3]{t})^{-4}$
      \item $\sqrt{\sqrt[3]{t^{-4}}}$
      \item $8^{\frac{2}{3}}$
      \item $9^{-\frac{1}{2}}$
      \item $(\sqrt[6]{49})^3$
      \item $\sqrt[5]{\sqrt[]{1024}}$
    \end{itemize}
  \end{Large}
\end{multicols}

\subsubsection*{グラフ}

\begin{table}[H]
  \begin{minipage}{0.5\linewidth}
    $y=3^x$
    \begin{center}
      \begin{tikzpicture}
        \draw[->] (-3,0) -- (3,0) node[right] {$x$};
        \draw[->] (0,-3) -- (0,3) node[above] {$y$};
      \end{tikzpicture}
    \end{center}
  \end{minipage}
  \begin{minipage}{0.5\linewidth}
    $y=(\frac{1}{3})^x$
    \begin{center}
      \begin{tikzpicture}
        \draw[->] (-3,0) -- (3,0) node[right] {$x$};
        \draw[->] (0,-3) -- (0,3) node[above] {$y$};
      \end{tikzpicture}
    \end{center}
  \end{minipage}
\end{table}

\subsection*{対数関数}
\begin{itembox}[l]{定義}
  \begin{multicols}{3}
    \begin{large}
      \begin{itemize}
        \item $\log_{10}100$
        \item $\log_{7}1$
        \item $\log_{6}6$
        \item $\log_{3}\frac{1}{9}$
        \item $\log_{2}\sqrt{32}$
      \end{itemize}
    \end{large}
  \end{multicols}
\end{itembox}

\begin{itembox}[l]{計算}
  \begin{multicols}{3}
    \begin{large}
      \begin{itemize}
        \item $\log_{4}8+\log_{4}2$
        \item $\log_{2}24-\frac{1}{2}\log_{2}9$
        \item $2\log_{2}27-\log_{2}9\log_{2}\sqrt{3}$
      \end{itemize}
    \end{large}
  \end{multicols}
\end{itembox}

\subsubsection*{底変換}
定義 $\log_{a}b$
\vspace{8mm}
\begin{itembox}[l]{例題}
  \begin{multicols}{3}
    \begin{large}
      \begin{enumerate}
        \item $\log_{9}27$
        \item $\log_{\frac{1}{2}}32$
        \item $\log_{8}2$
        \item $2\log_{3}6-\log_{9}16$
        \item $\log_{8}3\cdot\log_{9}25\cdot\log_{5}4$
      \end{enumerate}
    \end{large}
  \end{multicols}
\end{itembox}

\begin{itembox}[l]{対数関数の式の値}
  $a=\log_{10}2, b=\log_{10}3$
  \begin{multicols}{3}
    \begin{large}
      \begin{enumerate}
        \item $\log_{10}24$
        \item $\log_{10}5$
        \item $\log_{2}3$
      \end{enumerate}
    \end{large}
  \end{multicols}
\end{itembox}

\subsubsection*{グラフ}

\begin{table}[H]
  \begin{minipage}{0.5\linewidth}
    $y=\log_{2}x$
    \begin{center}
      \begin{tikzpicture}
        \draw[->] (-3,0) -- (3,0) node[right] {$x$};
        \draw[->] (0,-3) -- (0,3) node[above] {$y$};
      \end{tikzpicture}
    \end{center}
  \end{minipage}
  \begin{minipage}{0.5\linewidth}
    $y=\log_{\frac{1}{2}}x$
    \begin{center}
      \begin{tikzpicture}
        \draw[->] (-3,0) -- (3,0) node[right] {$x$};
        \draw[->] (0,-3) -- (0,3) node[above] {$y$};
      \end{tikzpicture}
    \end{center}
  \end{minipage}
\end{table}

\subsubsection*{常用対数}
\begin{itembox}[l]{例題}
  $\log_{10}2=0.3010, \log_{10}3=0.4771$
  \begin{large}
    \begin{enumerate}
      \item $2^{50}は何桁か$
      \item $0.3^{50}は小数第何位で初めて0でないか$
    \end{enumerate}
  \end{large}
\end{itembox}

\newpage
\subsection*{微分}
\subsubsection*{極限値}
\begin{itembox}[l]{例題}
  \begin{large}
    \begin{multicols}{2}
      \begin{enumerate}
        \item $\lim_{x\rightarrow2}(2x-1)$
        \item $\lim_{x\rightarrow-1}(3x^2+5x)$
      \end{enumerate}
    \end{multicols}
  \end{large}
\end{itembox}


\subsubsection*{微分の定義}
定義 $f(x)のx=aにおける微分係数$
\vspace{8mm}
\begin{itembox}[l]{例題}
  $f(x)=2x^2-3$の$x=2$における微分係数
\end{itembox}

\subsubsection*{導関数}
定義 $f(x)$
\vspace{8mm}
\begin{itembox}[l]{定義に従って導関数を求めよ}
  \begin{multicols}{2}
    \begin{large}
      \begin{enumerate}
        \item $f(x)=3x+1$
        \item $f(x)=2x^2$
      \end{enumerate}
    \end{large}
  \end{multicols}
\end{itembox}

\begin{itembox}[l]{微分せよ}
  \begin{large}
    \begin{multicols}{2}
      \begin{enumerate}
        \item $y=x^3-2x^2+5x-5$
        \item $y=(3x-1)^2$
      \end{enumerate}
    \end{multicols}
  \end{large}
\end{itembox}

\begin{itembox}[l]{微分係数とは何を表すか}
  \vspace{8mm}
\end{itembox}

\subsubsection*{接線}
\begin{itembox}[l]{例題}
  \begin{large}
    \begin{enumerate}
      \item $y=x^3-2x^2+5x+1上の点(2,11)における接線$
      \item $y=x^2-2x+3の接線で点(-1,-3)を通る接線$
      \item $y=-x^2+4x+3の傾きが6の接線$
    \end{enumerate}
  \end{large}
\end{itembox}


\subsubsection*{3次関数のグラフ}
\begin{itembox}[l]{微分と増減表、概形}
  \vspace{10mm}
\end{itembox}

\begin{itembox}[l]{例題}
  増減表とグラフの概形を書け
  \begin{large}
    \begin{enumerate}
      \item $y=x^3-6x^2+9x$\vspace{15mm}
      \item $y=-x^3+\frac{3}{2}x^2+6x-2$\vspace{15mm}
      \item $y=-2x^3+6x^2-6x+1$\vspace{15mm}
      \item $y=x^3+5x$\vspace{15mm}
    \end{enumerate}
  \end{large}
\end{itembox}

\subsubsection*{4次関数のグラフ}
\begin{itembox}[l]{例題}
  増減表とグラフの概形を書け
  \begin{large}
    \begin{enumerate}
      \item $y=x^4-4x^3+4x^2$\vspace{15mm}
      \item $y=-x^4+4x^3-5$\vspace{15mm}
    \end{enumerate}
  \end{large}
\end{itembox}


\newpage
\subsection*{積分}
\begin{itembox}[l]{積分とは}
  \vspace{10mm}
\end{itembox}
\subsubsection*{不定積分}
定義
\begin{LARGE}
  $\int x^ndx$
\end{LARGE}
\vspace{8mm}

\begin{itembox}[l]{例題}
  \begin{large}
    \begin{enumerate}
      \item $\int (3x^2+7x-3) dx$
      \item $\int (3x-2)^2dx$
    \end{enumerate}
  \end{large}
\end{itembox}

\subsubsection*{定積分}
性質
\begin{LARGE}
  \begin{itemize}
    \item $\int_{a}^{a}f(x)dx$
    \item $-\int_{a}^{b}f(x)dx$
    \item $\int_{a}^{c}f(x)dx+\int_{c}^{b}f(x)dx$
  \end{itemize}
\end{LARGE}

\begin{itembox}[l]{例題}
  \begin{large}
    \begin{enumerate}
      \item $\int_{-2}^{1}(x^2+5x-1)dx$
      \item $\int_{-1}^{2}(x-2)^2dx$
      \item $\int_{-1}^{1}x^2dx-\int_{2}^{1}x^2dx$
    \end{enumerate}
  \end{large}
\end{itembox}


\subsubsection*{面積}
\begin{itembox}[l]{例題}
  次の曲線とx軸で囲まれた面積
  \begin{large}
    \begin{enumerate}
      \item $y=x^2+x+2$
      \item $y=x^2-2x$
    \end{enumerate}
  \end{large}
\end{itembox}

\begin{itembox}[l]{例題}
  次の関数で囲まれた面積
  \begin{large}
    \begin{enumerate}
      \item $y=x^2+x-5, y=2x+1$
      \item $y=x^2+4x-5, y=-x^2-2x+3$
    \end{enumerate}
  \end{large}
\end{itembox}


\end{document}