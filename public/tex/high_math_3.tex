
\newif\ifanswer\answertrue
%\answerfalse

\documentclass[10pt,dvipdfmx]{jsarticle}
\usepackage[margin=15truemm]{geometry}
\usepackage[dvipdfmx]{graphicx}
\usepackage{wrapfig}
\usepackage[dvipdfmx]{color}
\usepackage{ascmac} % for scree
\usepackage{subfigure} % for subfigure
\usepackage{multicol}
\usepackage{setspace}
\usepackage{diagbox}
\usepackage{fancyhdr}
\usepackage{xcolor}
\usepackage{here}
\usepackage{amsmath}
\usepackage{amssymb}
\usepackage{tikz}
\usetikzlibrary{intersections, calc, arrows.meta}
\pagestyle{fancy}
\ifanswer
\lhead{数3 解答}
\else
\lhead{数3}
\fi
\rhead{\number\month\number\day}

\ifanswer
\newcommand{\answer}[2]{{\color{orange}#2}}
\newcommand{\page}[2]{#1}
\newcommand{\question}[2]{{\color{orange}#2}}
\else
\newcommand{\answer}[2]{\vspace{#1mm}}
\newcommand{\page}[2]{#2}
\newcommand{\question}[2]{#1}
\fi%answer

\begin{document}

\subsection*{極限}

\newpage
\subsection*{二次曲線}
\subsubsection*{放物線}
\begin{itembox}[l]{定義}
  \vspace{8mm}
\end{itembox}

\begin{multicols}{2}
  \begin{minipage}{0.45\textwidth}
    \begin{itembox}[l]{$x$軸が軸}
      標準形(焦点、準線):\\
      \begin{center}
        \begin{tikzpicture}
          \draw[->] (-3,0) -- (3,0) node[right] {$x$};
          \draw[->] (0,-3) -- (0,3) node[above] {$y$};
        \end{tikzpicture}
      \end{center}
    \end{itembox}

  \end{minipage}
  \begin{minipage}{0.45\textwidth}
    \begin{itembox}[l]{$y$軸が軸}
      標準形(焦点、準線):\\
      \begin{center}
        \begin{tikzpicture}
          \draw[->] (-3,0) -- (3,0) node[right] {$x$};
          \draw[->] (0,-3) -- (0,3) node[above] {$y$};
        \end{tikzpicture}
      \end{center}

    \end{itembox}
  \end{minipage}
\end{multicols}


\subsubsection*{楕円}
\begin{itembox}[l]{定義、標準形、焦点、長軸、短軸、円との関係}
  \vspace{20mm}
\end{itembox}

\begin{multicols}{2}
  \begin{minipage}{0.45\textwidth}
    \begin{itembox}[l]{\hspace{3cm}}
      \begin{center}
        \begin{tikzpicture}
          \draw[->] (-3,0) -- (3,0) node[right] {$x$};
          \draw[->] (0,-3) -- (0,3) node[above] {$y$};
        \end{tikzpicture}
      \end{center}
    \end{itembox}

  \end{minipage}
  \begin{minipage}{0.45\textwidth}
    \begin{itembox}[l]{\hspace{3cm}}
      \begin{center}
        \begin{tikzpicture}
          \draw[->] (-3,0) -- (3,0) node[right] {$x$};
          \draw[->] (0,-3) -- (0,3) node[above] {$y$};
        \end{tikzpicture}
      \end{center}
    \end{itembox}
  \end{minipage}
\end{multicols}


\subsubsection*{双曲線}
\begin{itembox}[l]{定義、標準形、焦点、漸近線}
  \vspace{20mm}
\end{itembox}
\begin{multicols}{2}
  \begin{minipage}{0.45\textwidth}
    \begin{itembox}[l]{\hspace{3cm}}
      \begin{center}
        \begin{tikzpicture}
          \draw[->] (-3,0) -- (3,0) node[right] {$x$};
          \draw[->] (0,-3) -- (0,3) node[above] {$y$};
        \end{tikzpicture}
      \end{center}
    \end{itembox}

  \end{minipage}
  \begin{minipage}{0.45\textwidth}
    \begin{itembox}[l]{\hspace{3cm}}
      \begin{center}
        \begin{tikzpicture}
          \draw[->] (-3,0) -- (3,0) node[right] {$x$};
          \draw[->] (0,-3) -- (0,3) node[above] {$y$};
        \end{tikzpicture}
      \end{center}
    \end{itembox}
  \end{minipage}
\end{multicols}


\subsubsection*{離心率}
\begin{itembox}[l]{定義}
  \vspace{8mm}
\end{itembox}
\begin{itemize}
  \item \item \item
\end{itemize}

\subsubsection*{極座標}

\begin{itembox}[l]{直行座標と極座標の関係}
  \begin{multicols}{2}
    \begin{minipage}{0.45\textwidth}
      $(x,y)と(r,\theta)$
    \end{minipage}
    \begin{minipage}{0.45\textwidth}
      \begin{tikzpicture}
        \draw[->] (-3,0) -- (3,0) node[right] {$x$};
        \draw[->] (0,-3) -- (0,3) node[above] {$y$};
      \end{tikzpicture}
    \end{minipage}


  \end{multicols}
\end{itembox}


\newpage
\subsection*{微分}
公式
\begin{large}
  \begin{enumerate}
    \item $y=x^n$
    \item $y=\cfrac{1}{x^n}$
    \item $y=f(x)g(x)$
    \item $y=\cfrac{f(x)}{g(x)}$
    \item $y=f(g(x))$
  \end{enumerate}
\end{large}

\begin{itembox}[l]{例題}
  \begin{large}
    \begin{enumerate}
      \item $y=\cfrac{1}{x\sqrt{x}}$
      \item $y=\sqrt{2x^2-3x}$
      \item $y=\cfrac{1}{\sqrt[3]{x^2-1}}$
    \end{enumerate}
  \end{large}
\end{itembox}

\begin{itembox}[l]{例題}
  $\cfrac{dy}{dx}$を$x,y$で表せ
  \begin{large}
    \begin{enumerate}
      \item $xy=3$
      \item $x^2+y^2=9$
    \end{enumerate}
  \end{large}
\end{itembox}

\begin{itembox}[l]{例題}
  $\cfrac{dy}{dx}$を$t$で表せ
  \begin{large}
    \begin{enumerate}
      \item $x=t+2, y=2t^2-3t$
      \item $x=\sqrt{t-1}, y=(3t-1)^2$
    \end{enumerate}
  \end{large}
\end{itembox}

三角関数の微分
\begin{itemize}
  \item $y=\sin x$
  \item $y=\cos x$
  \item $y=\tan x$
\end{itemize}

対数関数の微分
\begin{itemize}
  \item $y=\log_a x$
  \item $y=\log x$
\end{itemize}

\begin{itembox}[l]{例題}
  \begin{large}
    \begin{enumerate}
      \item $y=x^x$
    \end{enumerate}
  \end{large}
\end{itembox}

指数関数の微分
\begin{itemize}
  \item $y=a^x$
  \item $y=e^x$
\end{itemize}

\subsection*{積分}
\end{document}