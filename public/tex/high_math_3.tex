
\newif\ifanswer\answertrue
\answerfalse

\documentclass[10pt,dvipdfmx]{jsarticle}
\usepackage[margin=15truemm]{geometry}
\usepackage[dvipdfmx]{graphicx}
\usepackage{wrapfig}
\usepackage[dvipdfmx]{color}
\usepackage{ascmac} % for scree
\usepackage{subfigure} % for subfigure
\usepackage{multicol}
\usepackage{setspace}
\usepackage{diagbox}
\usepackage{fancyhdr}
\usepackage{xcolor}
\usepackage{here}
\usepackage{amsmath}
\usepackage{amssymb}
\usepackage{tikz}
\usetikzlibrary{intersections, calc, arrows.meta}
\pagestyle{fancy}
\ifanswer
\lhead{数3 解答}
\else
\lhead{数3}
\fi
\rhead{\number\month\number\day}

\ifanswer
\newcommand{\answer}[2]{{\color{orange}#2}}
\newcommand{\page}[2]{#1}
\newcommand{\question}[2]{{\color{orange}#2}}
\else
\newcommand{\answer}[2]{\vspace{#1mm}}
\newcommand{\page}[2]{#2}
\newcommand{\question}[2]{#1}
\fi%answer

\begin{document}

\subsection*{関数}
\subsubsection*{分数関数}
基本形、漸近線、定義域、値域
\begin{table}[h]
  \begin{minipage}{0.5\textwidth}
    \begin{tikzpicture}
      \draw[->] (-3,0) -- (3,0) node[right] {$x$};
      \draw[->] (0,-3) -- (0,3) node[above] {$y$};
    \end{tikzpicture}
  \end{minipage}
  \begin{minipage}{0.5\textwidth}
    \begin{tikzpicture}
      \draw[->] (-3,0) -- (3,0) node[right] {$x$};
      \draw[->] (0,-3) -- (0,3) node[above] {$y$};
    \end{tikzpicture}
  \end{minipage}
\end{table}

\subsubsection*{無理関数}
式、定義域、値域

\begin{tikzpicture}
  \draw[->] (-7,0) -- (7,0) node[right] {$x$};
  \draw[->] (0,-3) -- (0,3) node[above] {$y$};
\end{tikzpicture}

\subsubsection*{逆関数}
求め方
\begin{Large}
  \begin{itemize}
    \item \item \item
  \end{itemize}
\end{Large}
性質
\vspace{10mm}

\newpage
\subsection*{極限}

\subsubsection*{無限等比数列}
$\lim_{n\rightarrow\infty}r^n$
\begin{Large}
  \begin{itemize}
    \item \item \item \item
  \end{itemize}
\end{Large}
*$\lim_{n\rightarrow\infty}ar^n$の収束条件は?

\subsubsection*{無限等比級数}
$\sum_{k=1}^{\infty}ar^k$
\begin{Large}
  \begin{itemize}
    \item \item \item
  \end{itemize}
\end{Large}

\begin{itembox}[l]{不定形となる時}
  \begin{Large}
    \begin{itemize}
      \item \item \item
    \end{itemize}
  \end{Large}
\end{itembox}

\subsubsection*{片側極限}
\begin{itembox}[l]{ポイント}
  \vspace{15mm}
\end{itembox}

\subsubsection*{指数関数}
$\lim_{x\rightarrow\infty}a^x$, $\lim_{x\rightarrow-\infty}a^x$
\begin{Large}
  \begin{itemize}
    \item \item
  \end{itemize}
\end{Large}

\subsubsection*{対数関数}
$\lim_{n\rightarrow\infty}\log_{a}x$, $\lim_{n\rightarrow+0}\log_{a}x$
\begin{Large}
  \begin{itemize}
    \item \item
  \end{itemize}
\end{Large}

\subsubsection*{三角関数}
\begin{Large}
  \begin{itemize}
    \item $\lim_{x\rightarrow0}\cfrac{\sin x}{x}$
    \item $\lim_{x\rightarrow0}\cfrac{x}{\sin x}$
    \item $\lim_{x\rightarrow0}\cfrac{\tan x}{x}$
    \item $\lim_{x\rightarrow0}\cfrac{1-\cos x}{x^2}$
  \end{itemize}
\end{Large}

\begin{itembox}[l]{三角関数の極限2}
  \begin{Large}
    \begin{itemize}
      \item \item
    \end{itemize}
  \end{Large}
\end{itembox}

\subsubsection*{関数の点連続性}
関数$f(x)$が$x=a$で連続であるための条件
\vspace{15mm}

\subsubsection*{微分係数の利用で指数関数や対数関数}
\begin{Large}
  \begin{itemize}
    % \item $\lim_{x\rightarrow0}\cfrac{e^x-1}{x}$
    % \item $\lim_{x\rightarrow1}\cfrac{\log x}{x-1}$
    \item \item
  \end{itemize}
\end{Large}

\subsubsection*{自然対数の利用}
\begin{Large}
  \begin{itemize}
    \item
    \item
  \end{itemize}
\end{Large}

\subsubsection*{定積分の定義の利用}
\vspace{15mm}

\newpage
\subsection*{二次曲線}
\subsubsection*{放物線}
\begin{itembox}[l]{定義}
  \vspace{8mm}
\end{itembox}

\begin{multicols}{2}
  \begin{minipage}{0.45\textwidth}
    \begin{itembox}[l]{$x$軸が軸}
      標準形(焦点、準線):\\
      \begin{center}
        \begin{tikzpicture}
          \draw[->] (-3,0) -- (3,0) node[right] {$x$};
          \draw[->] (0,-3) -- (0,3) node[above] {$y$};
        \end{tikzpicture}
      \end{center}
    \end{itembox}

  \end{minipage}
  \begin{minipage}{0.45\textwidth}
    \begin{itembox}[l]{$y$軸が軸}
      標準形(焦点、準線):\\
      \begin{center}
        \begin{tikzpicture}
          \draw[->] (-3,0) -- (3,0) node[right] {$x$};
          \draw[->] (0,-3) -- (0,3) node[above] {$y$};
        \end{tikzpicture}
      \end{center}

    \end{itembox}
  \end{minipage}
\end{multicols}


\subsubsection*{楕円}
\begin{itembox}[l]{定義、標準形、焦点、長軸、短軸、円との関係}
  \vspace{20mm}
\end{itembox}

\begin{multicols}{2}
  \begin{minipage}{0.45\textwidth}
    \begin{itembox}[l]{\hspace{3cm}}
      \begin{center}
        \begin{tikzpicture}
          \draw[->] (-3,0) -- (3,0) node[right] {$x$};
          \draw[->] (0,-3) -- (0,3) node[above] {$y$};
        \end{tikzpicture}
      \end{center}
    \end{itembox}

  \end{minipage}
  \begin{minipage}{0.45\textwidth}
    \begin{itembox}[l]{\hspace{3cm}}
      \begin{center}
        \begin{tikzpicture}
          \draw[->] (-3,0) -- (3,0) node[right] {$x$};
          \draw[->] (0,-3) -- (0,3) node[above] {$y$};
        \end{tikzpicture}
      \end{center}
    \end{itembox}
  \end{minipage}
\end{multicols}


\subsubsection*{双曲線}
\begin{itembox}[l]{定義、標準形、焦点、漸近線}
  \vspace{20mm}
\end{itembox}
\begin{multicols}{2}
  \begin{minipage}{0.45\textwidth}
    \begin{itembox}[l]{\hspace{3cm}}
      \begin{center}
        \begin{tikzpicture}
          \draw[->] (-3,0) -- (3,0) node[right] {$x$};
          \draw[->] (0,-3) -- (0,3) node[above] {$y$};
        \end{tikzpicture}
      \end{center}
    \end{itembox}

  \end{minipage}
  \begin{minipage}{0.45\textwidth}
    \begin{itembox}[l]{\hspace{3cm}}
      \begin{center}
        \begin{tikzpicture}
          \draw[->] (-3,0) -- (3,0) node[right] {$x$};
          \draw[->] (0,-3) -- (0,3) node[above] {$y$};
        \end{tikzpicture}
      \end{center}
    \end{itembox}
  \end{minipage}
\end{multicols}


\subsubsection*{離心率}
\begin{itembox}[l]{定義}
  \vspace{8mm}
\end{itembox}
\begin{itemize}
  \item \item \item
\end{itemize}

\subsubsection*{極座標}

\begin{itembox}[l]{直行座標と極座標の関係}
  \begin{multicols}{2}
    \begin{minipage}{0.45\textwidth}
      $(x,y)と(r,\theta)$
    \end{minipage}
    \begin{minipage}{0.45\textwidth}
      \begin{tikzpicture}
        \draw[->] (-3,0) -- (3,0) node[right] {$x$};
        \draw[->] (0,-3) -- (0,3) node[above] {$y$};
      \end{tikzpicture}
    \end{minipage}


  \end{multicols}
\end{itembox}

\subsubsection*{媒介変数表示}
\begin{multicols}{2}
  \begin{description}
    \item[放物線$y^2=4px$]\vspace{30mm}
    \item[円$x^2+y^2=r^2$]\vspace{30mm}
    \item[円$(x-a)^2+(y-b)^2=r^2$]\vspace{30mm}
    \item[楕円$\cfrac{x^2}{a^2}+\cfrac{y^2}{b^2}=1$]\vspace{30mm}
    \item[双曲線$\cfrac{x^2}{a^2}-\cfrac{y^2}{b^2}=1$]\vspace{30mm}
    \item[サイクロイド]\vspace{30mm}
  \end{description}
\end{multicols}



\newpage
\subsection*{微分}
\begin{multicols}{2}
  \begin{Large}
    \begin{itemize}
      \item $y=x^n$
      \item $y=\cfrac{1}{x^n}$
    \end{itemize}
  \end{Large}
\end{multicols}
三角関数の微分
\begin{multicols}{2}
  \begin{Large}
    \begin{itemize}
      \item $y=\sin x$
      \item $y=\cos x$
      \item $y=\tan x$
      \item $y=\cfrac{1}{\tan x}$
    \end{itemize}
  \end{Large}
\end{multicols}

対数関数の微分
\begin{multicols}{2}
  \begin{Large}
    \begin{itemize}
      \item $y=\log_a x$
      \item $y=\log x$
    \end{itemize}
  \end{Large}
\end{multicols}

指数関数の微分
\begin{multicols}{2}
  \begin{Large}
    \begin{itemize}
      \item $y=a^x$
      \item $y=e^x$
    \end{itemize}
  \end{Large}
\end{multicols}

公式
\begin{multicols}{2}
  \begin{Large}
    \begin{itemize}
      \item $y=f(x)g(x)$
      \item $y=\cfrac{f(x)}{g(x)}$
      \item $y=\cfrac{1}{g(x)}$
      \item $y=f(g(x))$
    \end{itemize}
  \end{Large}
\end{multicols}

\begin{itembox}[l]{例題}
  \begin{large}
    \begin{enumerate}
      \item $y=\cfrac{1}{x\sqrt{x}}$
      \item $y=\sqrt{2x^2-3x}$
      \item $y=\cfrac{1}{\sqrt[3]{x^2-1}}$
    \end{enumerate}
  \end{large}
\end{itembox}

\begin{itembox}[l]{例題}
  $\cfrac{dy}{dx}$を$x,y$で表せ
  \begin{large}
    \begin{enumerate}
      \item $xy=3$
      \item $x^2+y^2=9$
    \end{enumerate}
  \end{large}
\end{itembox}

\begin{itembox}[l]{例題}
  $\cfrac{dy}{dx}$を$t$で表せ
  \begin{large}
    \begin{enumerate}
      \item $x=t+2, y=2t^2-3t$
      \item $x=\sqrt{t-1}, y=(3t-1)^2$
    \end{enumerate}
  \end{large}
\end{itembox}

\begin{itembox}[l]{例題}
  \begin{large}
    \begin{enumerate}
      \item $y=x^x$
    \end{enumerate}
  \end{large}
\end{itembox}


\newpage
\subsection*{積分}
\begin{multicols}{2}
  \begin{Large}
    \begin{itemize}
      \item $\int x^n dx$
      \item $\int \cfrac{1}{x^n} dx$
    \end{itemize}
  \end{Large}
\end{multicols}

三角関数
\begin{multicols}{2}
  \begin{Large}
    \begin{itemize}
      \item $\int \sin x dx$
      \item $\int \cfrac{1}{\cos^2 x} dx$
      \item $\int \cos x dx$
      \item $\int \cfrac{1}{\sin^2 x} dx$
    \end{itemize}
  \end{Large}
\end{multicols}
指数関数
\begin{multicols}{2}
  \begin{Large}
    \begin{itemize}
      \item $\int e^x dx$
      \item $\int a^x dx$
    \end{itemize}
  \end{Large}
\end{multicols}
対数関数
\begin{multicols}{2}
  \begin{Large}
    \begin{itemize}
      \item $\int \log_a x dx$
    \end{itemize}
  \end{Large}
\end{multicols}

\subsubsection*{三角関数の相互関係}
\begin{multicols}{3}
  \begin{Large}
    \begin{itemize}
      \item \item \item
    \end{itemize}
  \end{Large}
\end{multicols}

\subsubsection*{倍角}
\begin{multicols}{2}
  2倍角の公式
  \begin{Large}
    \begin{itemize}
      \item $\sin x\cos x$
      \item $\sin^2 x$
      \item $\cos^2 x$
    \end{itemize}
  \end{Large}

  3倍角の公式
  \begin{Large}
    \begin{itemize}
      \item $\sin^3 x$
      \item $\cos^3 x$
    \end{itemize}
  \end{Large}
\end{multicols}

\subsubsection*{積和の公式}
\begin{multicols}{2}
  \begin{Large}
    \begin{itemize}
      \item $\sin \alpha\sin \beta$
      \item $\cos \alpha\cos \beta$
      \item $\sin \alpha\cos \beta$
      \item $\cos \alpha\sin \beta$
    \end{itemize}
  \end{Large}
\end{multicols}

\subsubsection*{置換積分}
\begin{itembox}[l]{ポイント}
  \vspace{15mm}
\end{itembox}

\subsubsection*{部分積分}
\begin{Large}
  $\int f(x)g(x)' dx=$
\end{Large}


\begin{itembox}[l]{例題}
  \begin{large}
    \begin{enumerate}
      \item $\int \cfrac{x}{x^2+1}dx$\vspace{10mm}
      \item $\int x(x-1)^5dx$\vspace{10mm}
      \item $\int xe^{2x}dx$\vspace{10mm}
      \item $\int x\sin xdx$\vspace{10mm}
      \item $\int x^2\log xdx$\vspace{10mm}
      \item $\int \log(x+1)dx$\vspace{10mm}
      \item $\int (\log x)^2dx$\vspace{10mm}
      \item $\int e^{-x}\sin 2xdx$\vspace{10mm}
      \item $\int \cfrac{x^3+x^2-1}{x^2-1}dx$\vspace{10mm}
      \item $\int \cfrac{x-3}{x^2-3x+2}dx$\vspace{10mm}
      \item $\int \cfrac{x}{\sqrt{2x+3}-\sqrt{3}}dx$\vspace{10mm}
      \item $\int \sqrt{4-x^2}dx$\vspace{10mm}
      \item $\int \cfrac{1}{x^2+4}dx$\vspace{10mm}
    \end{enumerate}
  \end{large}
\end{itembox}

\end{document}