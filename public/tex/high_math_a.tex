\newif\ifanswer\answertrue
%\answerfalse

\documentclass[10pt,dvipdfmx]{jsarticle}
\usepackage[margin=15truemm]{geometry}
\usepackage[dvipdfmx]{graphicx}
\usepackage{wrapfig}
\usepackage[dvipdfmx]{color}
\usepackage{ascmac} % for scree
\usepackage{subfigure} % for subfigure
\usepackage{multicol}
\usepackage{setspace}
\usepackage{diagbox}
\usepackage{fancyhdr}
\usepackage{xcolor}
\usepackage{here}
\usepackage{amsmath}
\usepackage{amssymb}
\usepackage{tikz}
\usetikzlibrary{intersections, calc, arrows.meta}
\pagestyle{fancy}
\ifanswer
\lhead{数A 解答}
\else
\lhead{数A}
\fi
\rhead{\number\month\number\day}

\ifanswer
\newcommand{\answer}[2]{{\color{orange}#2}}
\newcommand{\page}[2]{#1}
\newcommand{\question}[2]{{\color{orange}#2}}
\else
\newcommand{\answer}[2]{\vspace{#1mm}}
\newcommand{\page}[2]{#2}
\newcommand{\question}[2]{#1}
\fi%answer

\begin{document}
\subsection*{順列と組合せ}
\subsubsection*{基本の計算}
\begin{itembox}[l]{3つの計算式 違い}
  \begin{multicols}{3}
    \begin{itemize}
      \item \item \item
    \end{itemize}
  \end{multicols}
\end{itembox}

\begin{itembox}[l]{約数の個数と展開式の項の個数と総和}
  \begin{enumerate}
    \item 200の正の約数の個数と総和を求めよ。
    \item 360の正の約数の個数と総和を求めよ。
  \end{enumerate}
\end{itembox}
\begin{itembox}[l]{文字の順列 a,b,c,d,eを1列に並べる}
  \begin{enumerate}
    \item a,bが隣り合う並べ方
    \item a,bが両端にくる並べ方
  \end{enumerate}
\end{itembox}
\begin{itembox}[l]{数字の順列数字の順列 0,1,2,3,4の5つの数字が1つずつある}
  \begin{enumerate}
    \item 3桁の整数
    \item 3桁の暗証番号
    \item 3桁の偶数
    \item 3桁の整数のうち、300以上の整数
  \end{enumerate}
\end{itembox}
\begin{itembox}[l]{円順列とじゅず順列}
  8種類の球を用いて次の場合の数を求めよ。
  \begin{enumerate}
    \item 円状に並べる方法
    \item じゅずを作るときの方法
  \end{enumerate}
\end{itembox}
\begin{itembox}[l]{条件付き円順列}
  先生2人と生徒4人が円形のテーブルに座るとき、次の場合の数を求めよ。
  \begin{enumerate}
    \item すべての座り方
    \item 先生2人が隣り合う座り方
    \item 先生2人が向い合う座り方
  \end{enumerate}
\end{itembox}
\begin{itembox}[l]{重複を許す順列}
  \begin{enumerate}
    \item a,b,c,d,e の5つの文字から、重複を許して3つの文字を一列に並べる並べ方
    \item 0 , 1 , 2 , 2 , 4 の5つの数字から、重複を許して3桁の自然数を作る作り方
  \end{enumerate}
\end{itembox}
\begin{itembox}[l]{2つのグループに分ける}
  9人を以下の方法で分ける場合の数を求めよ。
  \begin{enumerate}
    \item A、Bの2部屋に分ける方法(ただし、空室があってもよい)
    \item A、Bの2グループに分ける方法
    \item 2つのグループに分ける方法
  \end{enumerate}
\end{itembox}
\begin{itembox}[l]{順列と組合せ}
  a,b,c,d,e の5つの文字がそれぞれ1つずつあるとき、次の問いに答えよ。
  \begin{enumerate}
    \item 3つの文字を選び一列に並べるときの場合の数
    \item 3つの文字を選ぶときの場合の数
  \end{enumerate}
\end{itembox}
\begin{itembox}[l]{図形と組合せ}

  \begin{enumerate}
    \item 5本の平行線と、それとは別の3本の平行線とが交わってできる平行四辺形の数
    \item 正八角形について、頂点を結んでできる三角形の個数
    \item 正八角形について、頂点を結んでできる対角線の本数
  \end{enumerate}
\end{itembox}
\begin{itembox}[l]{代表を選ぶ}
  男子5人、女子4人から代表を3人選ぶ。このとき、次の場合の数を求めよ。
  \begin{enumerate}
    \item すべての選び方
    \item 男子1人、女子2人となる選び方
    \item 少なくとも女子1人を選ぶ選び方
    \item 男子から3人、または女子から3人を選ぶ選び方
  \end{enumerate}
\end{itembox}
\begin{itembox}[l]{3つのグループに分ける}
  9人を以下の方法で分ける場合の数を求めよ。
  \begin{enumerate}
    \item 3人ずつA、B、Cの3部屋に分ける
    \item 3人ずつ3組に分ける
    \item 4人、3人、2人に分ける
  \end{enumerate}
\end{itembox}
\begin{itembox}[l]{同じものを含む順列}
  \begin{enumerate}
    \item a,a,b,b,b,c,d の7つの文字を一列に並べる
    \item a,a,b,b,c,d,e の7つの文字を一列に並べるとき、c,d,e がこの順になる
  \end{enumerate}
\end{itembox}
\begin{itembox}[l]{最短経路問題}
  \begin{minipage}{0.7\textwidth}
    \begin{enumerate}
      \item AからBまでの最短経路
      \item AからBまでの最短経路でCを必ず通る経路
      \item AからBまでの最短経路でDを通らない経路
    \end{enumerate}
  \end{minipage}
  \begin{minipage}{0.2\textwidth}
    \begin{tikzpicture}[scale=0.7]
      \draw(0,0)grid(4,5);
      \coordinate (A) at (0,0) node [below left] at (A) {A};
      \coordinate (B) at (4,5) node [above right] at (B) {B};
      \coordinate (C) at (2,3) node [above left] at (C) {C};
      \coordinate (D) at (2.5,2) node [below] at (D) {D};
      \fill(0,0)circle(0.06)node[below]{};
      \fill(4,5)circle(0.06)node[below]{};
      \fill(2,3)circle(0.06)node[below]{};
      \fill(2.5,2)circle(0.06)node[below]{};
    \end{tikzpicture}
  \end{minipage}


\end{itembox}
\begin{itembox}[l]{重複組合せ}
  \begin{enumerate}
    \item 6本の同種類のペンをA、B、Cの3つの袋に入れるとき、1本も入らない袋があってよいとき、分け方は何通りあるか。
    \item オレンジ、レモン、ライムがそれぞれ多数ある。これから10個をまとめてセットを作りたい。何通りのセットができるか。
  \end{enumerate}
\end{itembox}

\begin{itembox}[l]{等式を満たす整数}
  \begin{enumerate}
    \item $x+y+z=10(x,y,z:0以上の整数)$の時の組合せ
    \item $x+y+z=10(x,y,z:自然数)$の時の組合せ
  \end{enumerate}
\end{itembox}

\subsection*{確率}
\begin{itembox}[l]{確率の基本}
  コインを3枚同時に投げるとき、次の確率を求めよ。
  \begin{enumerate}
    \item 2枚だけ表である確率
    \item 表が2枚以上である確率
  \end{enumerate}
\end{itembox}
\begin{itembox}[l]{さいころの確率}
  さいころを2個同時に投げるとき、次の確率を求めよ。
  \begin{enumerate}
    \item 目の和が8となる確率
    \item 目の和が10以下となる確率
  \end{enumerate}
\end{itembox}
\begin{itembox}[l]{ボールを取り出す確率}
  赤玉5個と白玉7個が入った袋から同時に3個取り出すとき、次の確率を求めよ。
  \begin{enumerate}
    \item 白玉3個となる確率
    \item 赤玉1個、白玉2個となる確率
    \item 赤玉2個、白玉1個となる確率
  \end{enumerate}
\end{itembox}

\begin{itembox}[l]{一列に並べる確率}
  男子5人、女子4人が1列に並ぶとき、次の確率を求めよ。
  \begin{enumerate}
    \item 特定の男女が隣り合う
    \item 女子が両端にいる
    \item 男女が交互に並ぶ
  \end{enumerate}
\end{itembox}
\begin{itembox}[l]{円形に並べる確率}
  男子3人、女子3人が円卓にする座るとき、次の確率を求めよ。
  \begin{enumerate}
    \item 特定の2人が隣り合う
    \item 特定の2人が向い合う
    \item 男女が交互に座る
  \end{enumerate}
\end{itembox}
\begin{itembox}[l]{和事象と排反事象}
  1~50までの数字が書かれたカードから、1枚取り出すとき、次の確率を求めよ。
  \begin{enumerate}
    \item 2の倍数または一の位が3である2桁の数
    \item 2の倍数または3の倍数
  \end{enumerate}
\end{itembox}
\begin{itembox}[l]{余事象の確率}
  \begin{enumerate}
    \item 赤玉5個と白玉7個が入った袋から同時に3個取り出すとき、少なくとも赤玉1個を取り出す確率を求めよ。
    \item さいころを2個同時に投げるとき、目の和が3の倍数でない確率を求めよ。
  \end{enumerate}
\end{itembox}
\begin{itembox}[l]{独立試行の確率}
  Aの袋には赤玉3個と白玉2個が、Bの袋には赤玉2個と白玉4個が入っている。Aからは1個、Bからは2個の玉を取り出すとき、取り出した玉の色がすべて赤となる確率を求めよ。
\end{itembox}
\begin{itembox}[l]{反復試行の確率(コイン)}
  1枚のコインを5回連続して投げるとき、次の確率を求めよ。
  \begin{enumerate}
    \item 表がちょうど4回出る
    \item 表がちょうど3回出る
  \end{enumerate}
\end{itembox}
\begin{itembox}[l]{反復試行の確率(さいころ)}
  1個のさいころを5回連続して投げるとき、次の確率を求めよ。
  \begin{enumerate}
    \item 3の倍数の目が2回だけ出る
    \item 3の倍数の目が3回だけ出る
    \item 少なくとも1回3の倍数の目が出る
  \end{enumerate}
\end{itembox}

\begin{itembox}[l]{勝先取の確率}
  AとBが試合をし、先に3勝した方が優勝とする。Aが勝つ確率が$\frac{3}{4}$のとき、Aが優勝する確率を求めよ。
\end{itembox}

\begin{itembox}[l]{点が動く確率}
  数直線上に点Pが原点にあり、さいころを投げて5以上の目が出ると正の方向に2進み、それ以外が出ると負の方向に1進む。さいころを3回投げたとき点Pが次の位置にある確率を求めよ。
  \begin{enumerate}
    \item 原点の位置にある
    \item 座標3の位置にある
  \end{enumerate}
\end{itembox}
\begin{itembox}[l]{条件付き確率}
  ある学校で数学が好きな生徒は40\%で、英語が好きな生徒は60\%で、両方好きな生徒は30\%である。
  \begin{enumerate}
    \item ある生徒が数学を好きとわかっていて、その生徒が英語も好きな確率
    \item ある生徒が英語を好きとわかっていて、その生徒が数学も好きな確率
  \end{enumerate}
\end{itembox}

\begin{itembox}[l]{確率の乗法定理}
  10本中当たりが3本入ったくじがある。このくじをAが1本引き、引いたくじを元に戻さずに続けてBが引いた。このとき、AとBのそれぞれが当たる確率を求めよ。
\end{itembox}

\newpage
\subsection*{図形}
\subsubsection*{内分と外分}
\begin{itembox}[l]{例題}
  \begin{enumerate}
    \item 線分ABを$3:1$で内分する点P
    \item 線分ABを$2:1$で外分する点Q
    \item 線分ABを$1:3$で外分する点R
  \end{enumerate}
  \begin{tikzpicture}[scale=0.5]
    \draw[->,>=stealth,semithick] (-5,0)--(10,0)node[above]{}; %x軸
    \draw (0,0.2)node[above]{A};
    \draw (4,0.2)node[above]{B};
    \foreach \x in {-5,...,10} \draw (\x,0)--(\x,0.2) node[above] {  };
  \end{tikzpicture}
\end{itembox}

\subsubsection*{角の二等分線}
\begin{minipage}[t]{0.5\textwidth}
  \begin{tikzpicture}
    %三角形の頂点を設定し,三角形を描く
    \coordinate (O) at (0,0) node [below] at (O) {};
    \coordinate (A) at (5,0) node [below] at (A) {};
    \coordinate (B) at (6,3) node [above right] at (B) {};
    \coordinate (C) at (9,4.5) node [above right] at (C) {};
    \coordinate (D) at (12,0) node [above right] at (D) {};
    \draw (O)--(A)--(B)--cycle;
    \draw[dashed](B)--(C);
    \draw[dashed](A)--(D);
    %各辺上で一定の長さの点を設定する
    \coordinate (s) at ($(O)!1cm!(A)$);
    \coordinate (t) at ($(O)!1cm!(B)$);
    \coordinate (u) at ($(A)!1cm!(O)$);
    \coordinate (v) at ($(A)!1cm!(B)$);
    \coordinate (w) at ($(B)!1cm!(O)$);
    \coordinate (z) at ($(B)!1cm!(A)$);
    \coordinate (c) at ($(B)!1cm!(C)$);
    %内角の二等分線
    \draw [dashed](B)--($(z)!.5!(w)!-4!(B)$) [name path=line B];
    \draw [dashed](B)--($(z)!.5!(c)!-13!(B)$) [name path=line C];
  \end{tikzpicture}
\end{minipage}

\begin{multicols}{2}
  \begin{minipage}{0.5\textwidth}
    \subsubsection*{三角形の外心}
    \vspace{60mm}
  \end{minipage}
  \begin{minipage}{0.5\textwidth}
    \subsubsection*{三角形の内心}
    \vspace{60mm}
  \end{minipage}
  \begin{minipage}{0.5\textwidth}
    \subsubsection*{三角形の垂心}
    \vspace{60mm}
  \end{minipage}
  \begin{minipage}{0.5\textwidth}
    \subsubsection*{三角形の重心}
    \vspace{60mm}
  \end{minipage}
\end{multicols}


\begin{multicols}{2}
  \begin{minipage}{0.5\textwidth}
    \subsubsection*{チェバの定理}
    \vspace{60mm}
  \end{minipage}
  \begin{minipage}{0.5\textwidth}
    \subsubsection*{メネラウスの定理}
    \vspace{60mm}
  \end{minipage}
\end{multicols}

\subsubsection*{円周角の定理}
\begin{multicols}{3}
  \begin{minipage}{0.33\textwidth}
    \begin{tikzpicture}
      \draw(0,0)circle(2.5);
      \fill(0,0)circle(0.06)node[below]{O};
    \end{tikzpicture}

  \end{minipage}
  \begin{minipage}{0.33\textwidth}
    \begin{tikzpicture}
      \draw(0,0)circle(2.5);
      \fill(0,0)circle(0.06)node[below]{O};
    \end{tikzpicture}
  \end{minipage}
  \begin{minipage}{0.33\textwidth}
    \begin{tikzpicture}
      \draw(0,0)circle(2.5);
      \fill(0,0)circle(0.06)node[below]{O};
    \end{tikzpicture}
  \end{minipage}
\end{multicols}

\begin{multicols}{2}
  \begin{minipage}{0.5\textwidth}
    \subsubsection*{円に内接する四角形}
    \begin{tikzpicture}
      \draw(0,0)circle(2.5);
      \fill(0,0)circle(0.06)node[below]{O};
    \end{tikzpicture}
  \end{minipage}
  \begin{minipage}{0.5\textwidth}
    \subsubsection*{接弦定理}
    \begin{tikzpicture}
      \draw(0,0)circle(2.5);
      \fill(0,0)circle(0.06)node[below]{O};
    \end{tikzpicture}
  \end{minipage}
\end{multicols}

\subsubsection*{方べきの定理}
\begin{multicols}{2}
  \begin{itemize}
    \item 接線でない場合\vspace{60mm}
    \item 接線の場合\vspace{60mm}
  \end{itemize}
\end{multicols}

\subsubsection*{円と接線の関係}
二つの円と共通接線の本数(5)
\vspace{10cm}


\newpage
\subsection*{整数}
\subsubsection*{用語}
\begin{itemize}
  \item 素数
  \item 互いに素
  \item 既約分数
\end{itemize}

\subsubsection*{倍数判定法}
\begin{multicols}{2}
  \begin{itemize}
    \item 2の倍数
    \item 3の倍数
    \item 4の倍数
    \item 5の倍数
    \item 8の倍数
    \item 9の倍数
  \end{itemize}
\end{multicols}

\subsubsection*{最小公倍数と最大公約数}
\begin{itembox}[l]{例題1}
  \begin{enumerate}
    \item \begin{multicols}{2}
            \begin{large}
              \begin{enumerate}
                \item 75, 105
                \item 42, 78, 273
              \end{enumerate}
            \end{large}
          \end{multicols}
    \item
          2つの自然数の最大公約数が 6、最小公倍数が 420 であるとき、この2つの自然数の組をすべて答えよ。
  \end{enumerate}
  \vspace{10mm}
\end{itembox}

\subsubsection*{ユーグリッドの互除法}
\begin{itembox}[l]{例題 最大公約数を求めろ}
  \begin{large}
    \begin{enumerate}
      \item 407,77
      \item 336, 180
    \end{enumerate}
  \end{large}
\end{itembox}

\subsubsection*{不定方程式}
\begin{itembox}[l]{例題}
  \begin{large}
    \begin{enumerate}
      \item $5x+2y=0$
      \item $5x+2y=1$
      \item $5x+2y=2$
      \item $44x+35y=1$
      \item $44x+35y=3$
    \end{enumerate}
  \end{large}
\end{itembox}

\subsubsection*{n進法}
\begin{itembox}[l]{例題}
  \begin{multicols}{2}
    \begin{large}
      \begin{enumerate}
        \item $11010_{(2)}$
        \item $2121_{(3)}$
        \item $3A_{(16)}$
        \item $38 [2]$
        \item $439 [5]$
        \item $91 [16]$
      \end{enumerate}
    \end{large}
  \end{multicols}
\end{itembox}
\begin{itembox}[l]{例題(小数)}
  \begin{multicols}{2}
    \begin{large}
      \begin{enumerate}
        \item $0.101_{(2)}$
        \item $11.231_{(5)}$
        \item $0.625[2]$
        \item $6.728[5]$
      \end{enumerate}
    \end{large}
  \end{multicols}
\end{itembox}

\subsubsection*{合同式}
定義 $a\equiv b \pmod m$
\begin{itembox}[l]{例題}
  \begin{large}
    \begin{enumerate}
      \item 15の50乗を7でわったあまりを求めろ
      \item $x+4\equiv 2\pmod 6$
      \item $3x\equiv 4 \pmod 5$
      \item $47^2011$の一の位
    \end{enumerate}
  \end{large}
\end{itembox}
\end{document}