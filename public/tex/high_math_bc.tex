\newif\ifanswer\answertrue
%\answerfalse

\documentclass[10pt,dvipdfmx]{jsarticle}
\usepackage[margin=15truemm]{geometry}
\usepackage[dvipdfmx]{graphicx}
\usepackage{wrapfig}
\usepackage[dvipdfmx]{color}
\usepackage{ascmac} % for scree
\usepackage{subfigure} % for subfigure
\usepackage{multicol}
\usepackage{setspace}
\usepackage{diagbox}
\usepackage{fancyhdr}
\usepackage{xcolor}
\usepackage{here}
\usepackage{amsmath}
\usepackage{amssymb}
\usepackage{tikz}
\usetikzlibrary{intersections, calc, arrows.meta}
\pagestyle{fancy}
\ifanswer
\lhead{数BC 解答}
\else
\lhead{数BC}
\fi
\rhead{\number\month\number\day}

\ifanswer
\newcommand{\answer}[2]{{\color{orange}#2}}
\newcommand{\page}[2]{#1}
\newcommand{\question}[2]{{\color{orange}#2}}
\else
\newcommand{\answer}[2]{\vspace{#1mm}}
\newcommand{\page}[2]{#2}
\newcommand{\question}[2]{#1}
\fi%answer

\begin{document}
\subsection*{数列}
\subsubsection*{等差数列}
\begin{itemize}
  \item 一般項(定数名も)
  \item 和
        \begin{itemize}
          \item 初項と末項がわかる
          \item 初項と末項がわからない
        \end{itemize}
\end{itemize}
\subsubsection*{等比数列}
\begin{itemize}
  \item 一般項(定数名も)
  \item 和
\end{itemize}

\subsubsection*{和の記号シグマ$\sum$}
\begin{multicols}{2}
  \begin{large}
    \begin{itemize}
      \item $\sum_{k=1}^{n}c$
      \item $\sum_{k=1}^{n}k$
      \item $\sum_{k=1}^{n}k^2$
      \item $\sum_{k=1}^{n}k^3$
      \item $\sum_{k=1}^{n}r^k$
    \end{itemize}
  \end{large}
\end{multicols}

\begin{itembox}[l]{例題}
  \begin{large}
    \begin{enumerate}
      \item $\sum_{k=1}^{n}k^3-3k^2+3^k$
    \end{enumerate}
  \end{large}
\end{itembox}

\subsubsection*{分数数列の和}
\begin{itembox}[l]{例題}
  \begin{Large}
    \begin{enumerate}
      \item $\sum_{k=1}^{n}\frac{1}{k(k+1)}$
      \item $\sum_{k=1}^{n}\frac{1}{(2k-1)(2k+1)}$
    \end{enumerate}
  \end{Large}
\end{itembox}

\subsubsection*{等差数列$\times$等比数列}
\begin{itembox}[l]{例題}
  $S=1+2\cdot2+3\cdot2^2+\dots+n\cdot2^{n-1}$
  \vspace{15mm}
\end{itembox}

\subsubsection*{階差数列}
数列$a_n$は階差$b_n$を持つとき。
\vspace{8mm}

\subsubsection*{漸化式}
\begin{multicols}{3}
  \begin{large}
    \begin{itemize}
      \item $a_{n+1}=a_n+d$
      \item $a_{n+1}=ra_n$
      \item $a_{n+1}=a_n+f(n)$
    \end{itemize}
  \end{large}
\end{multicols}
\vspace{8mm}
$a_{n+1}=pa_n+q$のとき




\newpage
\subsection*{ベクトル}
平行四辺形OACBにおいて$\vec{OA}=\vec{a}, \vec{OB}=\vec{b}$とする
\begin{multicols}{3}
  \begin{itemize}
    \item $\vec{OC}$
    \item $\vec{AB}$
    \item $\vec{AC}$
  \end{itemize}
\end{multicols}

\subsubsection*{ベクトルの内積}
\vspace{10mm}

\subsubsection*{三角形の面積}
\vspace{10mm}

\subsubsection*{内分, 外分}
2点A($\vec{a}$),B($\vec{b}$)を$m:n$に内分または外分する点
\vspace{10mm}

\subsubsection*{直線上にある}
2点A($\vec{a}$),B($\vec{b}$)を結ぶ直線上にある点P($\vec{p}$)
\vspace{10mm}


\newpage
\subsection*{複素数平面}
\subsubsection*{共役な複素数}
\begin{itemize}
  \item $z=\bar{z}$ならば
  \item $z=-\bar{z}かつz\not=0$ならば
\end{itemize}

\subsubsection*{複素数平面}
\begin{multicols}{2}
  \begin{itembox}[l]{$z=a+bi$を平面上に表せ}
    \begin{tikzpicture}
      \draw[->] (-1,0)--(2,0) node[right]{$x$};
      \draw[->] (0,-1)--(0,2) node[above]{$y$};
    \end{tikzpicture}

    $|z|=$\\
    $|z\bar{z}|=$
  \end{itembox}
  \begin{itembox}[l]{和・差・実数倍}
    $\alpha,\beta$に対して$\alpha+\beta, \alpha-\beta, k\alpha$\\
    \begin{tikzpicture}
      \draw[->] (-1,0)--(3,0) node[right]{$x$};
      \draw[->] (0,-1)--(0,3) node[above]{$y$};
      \draw[->] (0,0)--(2,1) node[above]{$\alpha$};
      \draw[->] (0,0)--(1,2) node[above]{$\beta$};
    \end{tikzpicture}
  \end{itembox}
\end{multicols}

\subsubsection*{極形式}
\begin{LARGE}
  $z=$\\
\end{LARGE}
$z_1=r_1(\cos\theta_1+i\sin\theta_1), z_2=r_2(\cos\theta_2+i\sin\theta_2)$の時(絶対値、偏角)
\begin{Large}
  \begin{itemize}
    \item $z_1z_2=$
    \item $\frac{z_1}{z_2}=$
  \end{itemize}
\end{Large}

\subsubsection*{ド・モアブルの定理}
\begin{large}
  $(\cos\theta+i\sin\theta)^n=$
\end{large}\\

\begin{itembox}[l]{例題}
  $z^6=1$
  \vspace{15mm}
\end{itembox}

\newpage
\begin{itembox}[l]{複素数平面と図形のポイント}
  \vspace{10mm}
\end{itembox}

\subsubsection*{点のまわりの回転}
\begin{large}
  \begin{itemize}
    \item $z$を原点中心に$\theta$回転させた点$w$
    \item $z_1$を中心に$z_2$を$\theta$回転させた点$\gamma$
  \end{itemize}
\end{large}

\subsubsection*{半直線のなす角}
\begin{table}[H]
  \begin{minipage}{0.7\textwidth}
    右図で、C($\gamma$)はB($\gamma$)を点A($\alpha$)を中心に$\theta$回転させて$k$倍した点とする。このとき成り立つ等式

    \vspace{15mm}

    また、$\angle BAC=$

    \vspace{10mm}

    A,B,Cが一直線上ならば

    \vspace{10mm}

    ABとACが垂直ならば

    \vspace{10mm}
  \end{minipage}
  \begin{minipage}{0.3\textwidth}
    \begin{tikzpicture}
      \draw[->] (-1,0)--(5,0) node[right]{$x$};
      \draw[->] (0,-1)--(0,5) node[above]{$y$};
      \draw (0.5,0.5) node[below]{$\alpha$};
      \draw[->] (0.5,0.5)--(3,1) node[above]{$\beta$};
      \draw[->] (0.5,0.5)--(2,4) node[above]{$\gamma$};
    \end{tikzpicture}

  \end{minipage}
\end{table}

\subsubsection*{複素数と図形}
\begin{itemize}
  \item 直線
        \begin{itemize}
          \item 原点と点mを結ぶ直線に平行で、点$\alpha$を通る直線\vspace{10mm}
          \item 原点と点mを結ぶ直線に垂直で、点$\alpha$を通る直線\vspace{10mm}
          \item 異なる点$\alpha,\beta$を通る直線上の点$z$\vspace{10mm}
        \end{itemize}
  \item 円
        \begin{itemize}
          \item 定点$\alpha$を中心とする半径rの円\vspace{10mm}
          \item 定点$\alpha, \beta$を直径とする円\vspace{10mm}
        \end{itemize}
\end{itemize}

\end{document}