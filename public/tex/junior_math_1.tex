\documentclass[10pt]{jsarticle}
\usepackage[margin=15truemm]{geometry}
\usepackage[dvipdfmx]{graphicx}
\usepackage{ascmac} % for screen
\usepackage{subfigure} % for subfigure
\usepackage{multicol}

\begin{document}
\section{計算}
\begin{itembox}[l]{分数の割り算のときはどうしますか?}
	.\\[10mm]
\end{itembox}
\begin{itembox}[l]{次の値を求めなさい}
	\begin{Large}
		\begin{itemize}
			\item $2^2$
			\item $(-3)^3$
		\end{itemize}
	\end{Large}
\end{itembox}
\begin{itembox}[l]{素数を列挙してください}
	.\\[10mm]
\end{itembox}
\begin{itembox}[l]{素因数分解してください}
	\begin{Large}
		\begin{itemize}
			\item $18$
			\item $64$
		\end{itemize}
	\end{Large}
\end{itembox}
\begin{itembox}[l]{次の方程式を解いてください}
	\begin{Large}
		\begin{itemize}
			\item $x+5=2$
			\item $3x+2=8$
			      \item$\frac{1}{3}x+5=2$
			\item $x:5=2:3$
		\end{itemize}
	\end{Large}
\end{itembox}



\section{関数}
\subsection{関数の一般式}
\begin{itembox}[l]{一般式をかけ}
	\begin{Large}
		\begin{itemize}
			\item 比例
			\item 反比例
		\end{itemize}
	\end{Large}
\end{itembox}



\section{図形}
\begin{itembox}[l]{次の線はどんな線ですか}
	\begin{Large}
		\begin{itemize}
			\item 垂直二等分線
			\item 角の二等分線
			\item 垂線
		\end{itemize}
	\end{Large}
\end{itembox}


\subsection{面積}
\begin{itembox}[l]{面積の公式}
	\begin{Large}
		\begin{itemize}
			\item 三角形
			\item ひし形
			\item 平行四辺形
			\item 円
			\item 扇型
		\end{itemize}
	\end{Large}
\end{itembox}

\subsection{体積}
\begin{itembox}[l]{体積の公式}
	\begin{Large}
		\begin{itemize}
			\item  \verb|~|柱
			\item  \verb|~|錐
			\item 球
		\end{itemize}
	\end{Large}
\end{itembox}

\subsection{表面積}
\begin{itembox}[l]{表面積の公式}
	\begin{Large}
		\begin{itemize}
			\item  球
			\item そのほかの立体
		\end{itemize}
	\end{Large}
\end{itembox}

\section{データ}
\begin{itembox}[l]{次の用語を説明せよ}
	\begin{Large}
		\begin{itemize}
			\item  階級
			\item 階級値
			\item 度数
			\item 度数分布表
			\item ヒストグラム
			\item 相対度数
			\item 累積〜
		\end{itemize}
	\end{Large}
\end{itembox}


\end{document}
